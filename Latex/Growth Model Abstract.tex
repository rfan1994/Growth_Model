\documentclass[12pt]{article}

\usepackage{amssymb,amsthm,amsmath}
\usepackage{mathtools} 
\usepackage{graphicx}
\usepackage{geometry,lscape}
\usepackage{longtable,multirow}
\usepackage{booktabs}
\usepackage{bbold}
\usepackage{tikz}

\setlength{\parindent}{0pt}
\setlength{\parskip}{1em}
\geometry{a4paper}


\title{Growth Model with Automation and Endogenous Human capital}
\date{}
\author{Rong Fan}

% A short description of your job market paper and how it fits into the current literature. 

\begin{document}
\maketitle
Will automation ultimately displace or complement skilled and unskilled workers if human capital accumulation endogenously responds to the technology change? How long does it take for skilled and unskilled workers to adapt to the new equilibrium? I study the effect of automation technology wave on the labor share, wage level and inequality, under the framework of task model with heterogeneous workers and endogenous human capital. Even if automation ultimately benefits all types of labor, the transition can be long, volatile and sub-optimal. 

The technology wave lowers the cost of automation, it increases automation, lowers labor share and raises wage inequality in the short run by direct productivity effect. However, the technology wave changes the demand for production factors and factor prices. In the long run, it could decrease the automation, raises labor share and lowers inequality when the price effect dominates. The significance of price effect depends on the substitutability between capital and labor and the comparative advantage of labor over machine on new developed tasks. 

Human capital of different skill groups respond to the technology wave differently, it amplifies the technology wave mainly through labor supply channel. I estimate the model by fitting the transition path to the data and discuss the optimal policy to optimize the transition. Empirical evidence is provided using Artificial Intelligence Occupational Impact (AIOI), estimated by Felten et al. (2019) and Occupational Information Network (O*NET). The development of AI increases the demand for social and technical skills, and workers respond to the technology change by increasing their content, process and social skills. 



\end{document}
