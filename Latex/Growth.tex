\documentclass[12pt]{article}

\usepackage{amssymb,amsthm,amsmath}
\usepackage{mathtools} 
\usepackage{graphicx}
\usepackage{geometry,lscape}
\usepackage{longtable,multirow}
\usepackage{booktabs}
\usepackage{bbold}
\usepackage{tikz}
\usepackage[font={small,bf}]{caption}
\usepackage[title]{appendix}
\usepackage{dcolumn}
\usepackage{hyperref}
\newcolumntype{d}[1]{D{.}{.}{#1}}

\linepenalty=10
\captionsetup[table]{skip=5pt}
\geometry{a4paper}

\newtheorem{assumption}{Assumption}
\newtheorem{proposition}{Proposition}


\date{}
\author{Rong Fan}

\begin{document}

\begin{titlepage}
\title{Interaction between Automation and Human Capital: Labor Share and Inequality}
\author{Rong Fan\thanks{I thank Bulent Guler, Todd Walker, Rupal Kamdar and Laura Liu for their thorough guidance with this project. For their suggestions and support I thank Pascual Restrepo, Eric Sims, Yongseok Shin, Jing Cynthia Wu and Choongryul Yang. All errors are my own.} 
\\
\\
\\ \href{https://rfan1994.github.io/files/Growth.pdf}{https://rfan1994.github.io/files/Growth.pdf}
\\ \href{https://rfan1994.github.io/files/Growth.pdf}{Click here for newest version}}
\date{\today}
\maketitle

\begin{abstract}
\noindent This paper studies the interaction between human capital and automation since human capital is essential for understanding how labor share, wage premium, and inequality change in the era of automation. I develop a theoretical model by introducing heterogeneous workers (skilled and unskilled) and endogenous human capital to a task model framework. I calibrate the model to fit the data between 1980 and 2005 and discuss the policy implications. First, I find that human capital and automation are racing against each other. Human capital accumulation decreases the automation level and increases the labor share by 0.33\%. Second, uneven responses of skilled and unskilled workers amplify inequality, explaining 77\% of the wage premium increase. Industry and occupation-level data confirm the implications of the model. Automation has a positive effect on overall skill levels and training investment. The responses of human capital to automation of skilled and unskilled workers are significantly different.  \\
\vspace{0in}\\
\noindent\textbf{Keywords:} Growth, Technology, Human capital, Automation, Inequality\\
\vspace{0in}\\
\noindent\textbf{JEL Classification: } J24, J31, O11, O14\\

\bigskip
\end{abstract}
\setcounter{page}{0}
\thispagestyle{empty}
\end{titlepage}
\pagebreak \newpage

\section{Introduction}
Automation and artificial intelligence (AI) are important to explain the simultaneous decrease in the labor share and increase in the college premium since 1980 (e.g., Acemoglu and Autor, 2011\nocite{AcemogluAutor2011}; Autor and Salomons, 2018\nocite{AutorSalomons2018}; Acemoglu and Restrepo, 2020\nocite{AcemogluRestrepo2020}; Aghion et al., 2021\nocite{Aghionetal2021}). Automation allows the machine to replace unskilled workers (high school graduates), reducing the labor share. However, it complements skilled workers (college graduates), increasing the college premium. 

Human capital accumulation is a key way that workers could respond to automation that has not been discussed much in the literature. There are two ways that workers can adapt their human capital. Workers could respond to automation by increasing the supply of skilled workers (extensive margin). The share of employment with a Bachelor's degree and higher has increased from 26.98\% to 43.74\% from 1992 to 2021.\footnote{Data source: FRED (Federal Reserve Economic Data)} Additionally, workers can increase their skill levels through on-the-job training (intensive margin). The ratio between training (e.g., GED, ESL, work-related courses) and working hours has increased from 0.244\% to 1.049\% for high school graduates and from 0.758\% to 1.212\% for college graduates from 1995 to 2005.\footnote{Data source: ATES (Adult Training and Education Survey)} Acquisition of new skills allows unskilled workers to relocate to less automated sectors and allows skilled workers to better take advantage of technology. 

Important questions are how the human capital of each skill group responds to automation and to what extent the response of human capital changes the effect of automation on the labor market (labor share and wage premium) and welfare inequality. Sachs and Kotlikoff (2012)\nocite{SachsKotlikoff2012} provide an earlier related study, which shows that young, low-skill workers respond to automation by decreasing investment in education as the wage depreciation limits their ability to save. Prettner and Strulik (2020)\nocite{PrettnerStrulik2020} show that automation increases the share of skilled workers because workers are motivated by higher wage premiums. In contrast to the existing studies that focus only on education choices, I study lifetime human capital accumulation, including on-the-job training. Athreya and Eberly (2015)\nocite{AthreyaEberly2015} document the stagnated college attainment accompanied by a substantial increase in the college premium. Not everyone is able or willing to attend college due to credit constraints or uncertainty, but most people spend a lot of time acquiring human capital throughout their life.

To study the interaction between automation and human capital and their macroeconomic implications, I adopt the task model framework of Acemoglu and Restrepo (2018)\nocite{AcemogluRestrepo2018}. Tasks are defined as units of work activity that produce output; they can be produced by machine (only if automated) or labor. I extend the model by introducing heterogeneous workers, endogenous human capital, and directed research and development (R\&D) sector. Workers are different in their skill levels (skilled and unskilled); skilled workers have a comparative advantage over newer tasks and are more able to adjust their human capital. I endogenize human capital following Grossman et al. (2021)\nocite{Grossmanetal2021}. Human capital is accumulated by learning at the opportunity cost of wage income but increases the productivity of workers. Research and development (R\&D) is directed and endogenous; skilled workers can become scientists and work in automation (automate old tasks favoring machine) or innovation (innovate new tasks favoring labor) sectors.

I start by studying the short-run effects of automation and human capital in a static model. Automation directly displaces unskilled workers (displacement effect) and indirectly affects skilled workers since unskilled workers relocate after being displaced (relocation effect). Higher automation increases the wage premium since it pushes unskilled workers to tasks that they have disadvantages compared with skilled workers. Higher human capital increases the effective labor supply, decreases automation incentives, and increases labor share. The human capital gap between skilled and unskilled workers increases the wage premium. 

The short-run effects are only part of the story. I then embed this framework in a dynamic environment where I allow physical and human capital accumulation and endogenous R\&D. The research is directed, and the R\&D sector decides at which rate to automate existing or innovate new tasks based on the profit and cost of each patent type. The increase of automation is endogenous, resulting from technological revolution -- a reduction in automation cost. After a reduction in automation cost, more scientists work in the automation sector and automation rate increases. Automation level increases and leads to higher innovation and human capital investments. A higher automation level complements the innovation sector since it drives up the capital price, pushes down the wage level, and increases the innovation patent profit. Innovation rate then increases and complements human capital investment since it increases future wage growth and incentivizes human capital investment. Higher human capital growth complements innovation and substitutes automation, which decreases the long-run automation level and increases the long-run labor share. 

I highlight the uneven impacts of automation on skilled and unskilled workers and the distinctive responses of skilled and unskilled workers to automation. First, skilled workers benefit more from the productivity effect (improvement of allocation and higher innovation rate) of the technological revolution, while unskilled workers are affected more by the displacement effect. As a result, skilled workers are more incentivized than unskilled workers to invest in their human capital to maximize the benefit of automation. Unskilled workers save more through physical capital and less through human capital to insure against the wage drop caused by automation. Second, skilled and unskilled workers accumulate their human capital in different ways. Skilled workers acquire skills through learning-or-doing, making it easier to adjust their human capital in response to the technology change. On the other hand, unskilled workers accumulate human capital passively through learning-by-doing, adapting to the technology change more slowly.  

Before calibrating the model and implementing counterfactual policy experiments, I solve the social planner's problem. There are two externalities in the decentralized economy. First, skilled workers overinvest in their human capital. Skilled workers don't internalize their negative externalities in the R\&D sector. High productivity in the production sector increases the opportunity cost of research and development. Second, too many tasks are automated. When developing innovation patents, the R\&D sector does not internalize the positive externalities they generate on existing tasks. Keeping that in mind, an automation tax and a training tax on skilled workers seem to be appropriate for redistributive fiscal policy. 

I calibrate the model to match key macroeconomic variables from 1980 (pre technological revolution) to 2005 (post technological revolution). The technological revolution is modeled as a reduction in automation cost to target the change in labor share. Human capital accumulation increases the labor share by about 0.33\% (0.82\% from skilled labor share and -0.48\% from unskilled labor share). The direct effect of automation contributes only 23.4\% of the wage premium increase, while the enlarged human capital gap causes 76.6\% of the increase. I simulate the transition numerically and implement counterfactual policy experiments with an automation tax or a training tax on skilled workers. After the reduction in automation cost, the R\&D sector shifts to the automation sector, more old tasks are automated, and fewer new tasks are invented. More physical are accumulated at this stage, and the capital deepening decreases the return to physical capital. Workers increase their human capital accumulation and reduce their labor supply. Automation continues to increase, and labor share falls further. The value of innovation patents increases as the automation level increases, and innovation starts to catch up with automation. Workers begin to supply more labor as more new jobs are created, and more jobs are created as the workers supply more labor with higher quality; labor share starts to recover. An automation tax decreases automation, increases labor share, and decreases wage inequality. A training tax increases automation and decreases labor share but is more efficient in reducing wage and welfare inequality. A training tax also increases the technology growth rate and accelerates the transition by lowering the opportunity cost of research. 

My research delivers four key results. First, a reduction in automation costs increases productivity. It improves production factor allocation and increases the innovation rate. The innovation sector develops more new tasks because of productivity gain. Second, human capital investment complements innovation but substitutes automation. Human capital investment increases effective labor supply and slows capital deepening. It decreases the overall automation level and increases labor share. With a high enough human capital growth rate, the human capital supply is abundant; the economy will never converge to a balanced growth path (BGP) where all the tasks are automated. Third, endogenous human capital amplifies the effect of the technological revolution. The direct effect of automation can explain only 23.4\% of the wage premium change; the rest of the change is attributed to the enlarged human capital gap. Skilled workers benefit more from automation and are more incentivized than unskilled workers to invest in their human capital; they also have a higher ability to adjust their human capital accumulation. As a result, the human capital gap between skilled and unskilled workers increases, leading to a higher wage premium. When the human capital is endogenous, the labor share drops more in the short run; and it takes longer for the innovation rate to catch up since workers spend extra time on training to adapt to the new technology environment and supply less labor. Fourth, an automation tax decreases automation and increases labor share. A training tax on skilled workers is more efficient in reducing inequality and accelerating the recovery. 

Last but not least, I confirm the empirical implications of the model using both industry and occupation level data. Using EU KLEMS (EU level analysis of capital, labor, energy, materials and service inputs) data, I find a positive response of education level to the development of automation, but with a four-year lag. Occupational level evidence is provided using O*NET (Occupational Information Network) and ATES (Adult Training and Education Survey). Skill levels grow faster in the occupations more exposed to automation, contributed mainly by the skilled workers. Skill level growth or training time is decreasing in the occupational automation exposure level but increasing in the workers' education level. Skilled and unskilled workers respond differently to automation; higher exposure to automation increases the human capital investment of skilled workers more than that of unskilled workers. 

This paper makes four contributions to the literature. First, this paper is closely related to the study of the effects of automation on labor share, employment, wage, and inequality. Acemoglu and Autor (2011)\nocite{AcemogluAutor2011} and Acemoglu and Restrepo (2018)\nocite{AcemogluRestrepo2018} develop theoretical frameworks to explore the displacement effect of automation. Many empirical studies support the theoretical predictions and show that automation is important in explaining the ongoing change in the US labor market. Automation decreases the labor share (Autor and Salomons, 2018\nocite{AutorSalomons2018}; Dinlersoz et al., 2018\nocite{Dinlersozetal2018}) and aggregate employment at the employment zone level (Aghioen et al., 2021\nocite{Aghioenetal2021}). In terms of inequality, automation can explain between 50\% and 70\% of changes in the US wage structure (Acemoglu and Restrepo, 2020\nocite{AcemogluRestrepo2020}) and increases welfare inequality through the return to wealth (Moll et al., 2021\nocite{Molletal2021}). I contribute to this literature by distinguishing the direct and indirect effects of automation through displacement and human capital response channels. 

Second, my research contributes to the literature on endogenous growth\footnote{Aghion et al. (2014)\nocite{Aghionetal2014} discuss the setups and applications of Schumpeterian growth theory. Innovations replace older technologies, resulting in endogenous growth; individuals can choose to allocate the labor supply between production and research.} and technology adoption. This literature goes back to Romer (1990)\nocite{Romer1990} and has developed into an integrated framework for understanding endogenous growth from macro and microeconomic perspectives (Acemoglu, 2009\nocite{Acemoglu2012}; Aghion and Howitt, 2008\nocite{AghionHowitt2008}). The setups of the research and development sector in my model are mostly similar to Acemoglu et al. (2013)\nocite{Acemogluetal2013}. In their model, unskilled workers are used in production, while skilled workers perform R\&D functions and operations. In contrast to the canonical growth model, the R\&D in my paper is directed; resources can be assigned to automation or innovation. Theory and data suggest that new technologies are complementary with skilled labor (Violante, 2008\nocite{Violante2008}; Sanders and Ter Weel\nocite{SandersTerWeel2000}). I suggest that human capital is an important source of skill bias. Benhabib et al. (2017)\nocite{Benhabibetal2017} argue that the innovation incentives and the growth rate depends on the adoption environment (Hall and Khan, 2003\nocite{HallKhan2003}; Atkeson and Kehoe, 2007\nocite{AtkesonKehoe2007}; and Acemoglu et al., 2013\nocite{Acemogluetal2013}). In this paper, I study how the adoption environment affects the intensity and direction of research, determining the automation level and the growth rate. 

Third, this paper contributes to the literature on human capital accumulation. I explore the heterogeneous human capital investment for different skill groups facing automation. The accumulation of human capital takes two forms. First, knowledge is passed across generations through education, and the initial skill type of the workers is determined by their choice of schooling investment (Schultz, 2003\nocite{Schultz2003}; Bohacek and Kapicka, 2008\nocite{BohacekKapicka2008}). Second, human capital can be accumulated after school. Experience can be accumulated by working through learning-by-doing (Imai et al., 2004\nocite{Imaietal2004}; Thompson, 2010\nocite{Thompson2010}; Burdett et al., 2011\nocite{Burdettetal2011}). Skills can also be acquired by training at the opportunity cost of wage income through learning-or-doing (Frazis, 2007\nocite{Frazis2007}). This paper focuses mainly on after-school human capital acquisition following Stantcheva (2015)\nocite{Stantcheva2015}. In my model, agents optimally choose the working and training time, and human capital evolves depending on the amount of training. I highlight the heterogeneous learning abilities of different types of workers.

Last, my paper explores the interaction between human capital and technology but distinguishes automation (substitutes labor) and innovation (complements labor) as two types of technological change. Technology and human capital can be twin engines of growth but can also be opponents racing against each other. The dynamics between innovation and human capital are complementary. Skills are required to implement and invent new technology (Lloyd-Ellis and Roberts, 2002\nocite{Lloyd-EllisRoberts2002}); new technology is more productive and will be adopted only when used with a high fraction of skilled workers (Beaudry et al., 2006\nocite{Beaudryetal2006}); the endogenous growth would happen only if both innovation and human capital grow at the same time (Stokey, 2014\nocite{Stokey2014}; Stokey, 2020\nocite{Stokey2020}). However, the dynamics between automation and human capital are competitive. Since workers and R\&D sectors are making decisions separately, growth can be described as a strategic game between workers and entrepreneurs on decisions of human capital and R\&D investment (Redding, 1996\nocite{Redding1996}).

In the rest of this paper, I start by setting up the model and solving static equilibrium in section 2. Then I bring the static framework under a dynamic environment in section 3. I solve the social planner's problem in section 4 to explore the appropriate taxes. Then I calibrate the model, solve the transition and implement counterfactual experiments in section 5. Section 6 provides empirical evidence for the key implications of the model. Section 7 concludes. 

\section{Model with Automation}
I develop a growth model with endogenous innovation, automation and human capital. I adopt the task model framework developed by Acemoglu and Restrepo (2018)\nocite{AcemogluRestrepo2018}. However, I focus on the directed research and development and endogenous human capital investment. Human capital growth is endogenized following Grossman et al. (2021)\nocite{Grossmanetal2021}, but with more generalized accumulation functions. 

\subsection{Environment}
Figure \ref{model} presents the structure of the model. The formal model of the economy has three sectors: household, production, and the research and development (R\&D) sector. There are two types of households: unskilled and skilled. Households consume, save and invest in human capital. Final good producers produce consumption goods by aggregating tasks produced by task producers. Tasks are defined as units of work activity that produce output. To produce tasks, task producers purchase patent intermediates (innovation or automation) from the R\&D sector and combine them with production factors (capital, low-skill or high-skill labor). Unskilled workers can only work in the production sector, supplying low-skill labor. Skilled workers can work both in the production and the R\&D sector, supplying high-skill labor or working as scientists to develop innovation or automation patents. 

\begin{figure}[h!]
\includegraphics[width = \textwidth]{Model}
\caption{Model Structure}
\label{model}
\end{figure}

The economy is populated by two types of representative households: skilled households and unskilled households. They have the same preference and discount rate, but supply different types of labor and have different learning abilities. The households make two decisions simultaneously. First, they make consumption-saving decisions to maximize lifetime utility. Second, they allocate their time between working and training to optimize their lifetime wage income. Unskilled households supply low-skill labor, and skilled household supply high-skill labor or work as scientists. Physical and human capital are two alternative ways for households to save for the future. If the return to physical capital (interest rate) is higher, households will prefer to save through physical capital; if the return to human capital (wage growth rate) is higher, they will favor human capital investment. 

There are task and final good producers in the production sector, and all the producers are perfectly competitive. To produce tasks, task producers purchase patent intermediates from the R\&D sector and combine them with production factors. If the task has been automated, the producers of this task will purchase automation patent intermediates and choose between labor (low and high-skill) and machines to minimize the production cost. If the task hasn't been automated, the producers of this task will purchase innovation patent intermediates and can only choose between low and high-skill labor to produce the intermediate goods. The final good producers purchase the tasks from the task producers at competitive prices and aggregate them into the consumption goods. 

The skilled workers can work as scientists in the R\&D sector to develop new patents. There are two types of patents: innovation patents bring new tasks to the economy, which increases labor productivity and can only be produced by labor; automaton patents allow the firm to use machine instead of labor to produce the task. After developing the patent, the scientists hold the patent right and sell the task-specific patent intermediates to task producers at the price $\psi$. The growth of the economy is endogenous and Schumpeterian. The old tasks are destroyed when new ones are introduced, keeping the task measure constant.\footnote{The task productivity is increasing in the task index. If the variety growth is not excluded, the growth rate will increase over time and converge to infinity.} 

\subsubsection*{Household}
The economy is populated by two types of representative households: skilled households with measure $\epsilon_H$ and unskilled households with measure $\epsilon_L$. They share the same constant relative risk aversion (CRRA) preference $\frac{C^{1-\theta}}{1-\theta}$, with $\theta$ being the intertemporal elasticity of substitution, and discount rate $\rho$. However, they supply different types of labor and have different learning abilities. Households maximize their lifetime utility by making consumption-saving and working-learning decisions. 

Unskilled households can only work in the production sector and supply low-skill labor. Each period, they receive return to capital $K_L$ at rate $r$, wage income $W_L$, and transfers from the government $T_L$. Given the current technological environment, they allocate their time $l_L$ to working and $1-l_L$ to training. The total low-skill labor supply is $\epsilon_Ll_L$. By allocating time to learning, they can increase the future wage by increasing their human capital, but at the opportunity cost of wage income. The accumulation of human capital is increasing in the time spent on training. Unskilled households maximize their lifetime utility: 
\begin{align*}
\rho V_L(K_L,h_L) = \max_{C_L,l_L} \quad \frac{C_L^{1-\theta}}{1-\theta}+V_{LK}(K_L,h_L)\dot{K}_L+V_{Lh}(K_L,h_L)\dot{h}_L,
\end{align*}
subject to the laws of motion of physical and human capital: 
\begin{align*}
\dot{K}_L &=r K_L+W_Ll_L-\tau_{hL}W_L(1-l_L)+T_L-C_L, \\
\dot{h}_L &= \frac{(1-l_L)^{\alpha_L}}{\mu_{hL}}.
\end{align*}

Skilled households also receive the return to capital $K_H$ at rate $r$, and transfers from the government $T_H$. They are different from unskilled workers in that they can work in the production sector by supplying high-skill labor ($L_H$) or the R\&D sector by supplying research labor for automation ($\epsilon_I$) and innovation ($\epsilon_N$). When they work in the production sector, they receive the wage income $W_H$. When they work in the research sector, they hold the patent they develop and receive the patent profits $\Pi$ by selling patent intermediates to task producers. Skilled households maximize their lifetime utility: 
\begin{align*}
\rho V_H(K_H,h_H) = \max_{C_H,l_H} \quad \frac{C_H^{1-\theta}}{1-\theta}+V_{HK}(K_H,h_H)\dot{K}_H+V_{Hh}(K_H,h_H)\dot{h}_H,
\end{align*}
subject to the low of motion of physical and human capital: 
\begin{align*}
\dot{K}_H &= r_t K_H +W_H l_H+\Pi-\tau_{hH}W_H(1-l_H)+T_H-C_H, \\
\dot{h}_H &= \frac{(1-l_H)^{\alpha_H}}{\mu_{hH}}.
\end{align*}
The government collects taxes on training at rates $\tau_{hL}$ and $\tau_{hH}$ for unskilled and skilled workers. Training tax is linear in the opportunity cost of training $W_L(1-l_L)$ and $W_H(1-l_H)$. 

The law of motion for human capital is a generalized form of Grossman et al. (2021)\nocite{Grossmanetal2021}. $\mu_{hL}$ and $\mu_{hH}$ capture the inverse of training efficiency, and $\alpha_L$ and $\alpha_H$ represent the training style. The training efficiency is decreasing in $\mu_{hj}, j\in\{L,H\}$, and the human capital growth converges to 0 when $\mu_{hj}$ converges to infinity. When $\alpha_j \to 0$, human capital is accumulated through learning-by-doing, and the human capital growth rate is constant, no matter how much time the workers spend on training. When $\alpha_j \to 1$, human capital is accumulated through learning-or-doing, and the human capital accumulation rate is linear in the time spent on training\footnote{Grossman et al. (2021) is a special case: $\mu = 1$ and $\alpha=1$}. Here I assume that $0\leq\alpha_L<\alpha_H\leq1$, skilled workers can adjust their human capital accumulation more easily through on-the-job training than unskilled workers, and unskilled workers mainly accumulate their human capital passively through working experience. 

\subsubsection*{Production Sector}

The final good sector is perfectly competitive and produces the consumption good $Y$ by combining a unit measure of tasks $i\in [N-1,N]$, with an elasticity of substitution $\sigma \in (0,\infty)$. The output of each task $i$ is $y(i)$. The creation of new tasks will destroy the oldest tasks, so the total measure of tasks in this economy remains constant at $1$. The production function of final good sector takes the form:
\begin{align*}
Y = \tilde{A}\Big(\int_{N-1}^{N}y(i)^{\frac{\sigma-1}{\sigma}}di\Big)^{\frac{\sigma}{\sigma-1}}.
\end{align*}
The competitive final good producers purchase task $i$ from the task producer at price $p(i)$, and solve the following problem:
\begin{align*}
\max \quad & Y-\int_{N-1}^Np(i)y(i)di. 
\end{align*}

The task sector is also competitive; task producers purchase task-specific intermediates $q(i)$ at price $\psi$ from patent holders, which embodies the technology. Then they combine the task-specific intermediates with production factors (capital or labor). Suppose the automation patent has not been developed for the task. In that case, the task producers can only purchase the innovation patent intermediates. They can only use labor (high-skill $h(i)$ or low-skill $l(i)$) to produce the intermediate goods $y(i)$. If the task has been automated, the task producer can choose between automation and innovation patents and combine capital $k(i)$ with automation patent or labor (high-skill $h(i)$ or low-skill $l(i)$) with innovation patent. The production function of task producers is Cobb-Douglas:
\begin{align*}
y(i) &= 
\begin{dcases}
q(i)^{\eta}\Big(k(i) + \gamma_L(i,h_L)l(i)+\gamma_H(i,h_H)h(i)\Big)^{1-\eta} , &\quad \text{$i$ is automated}  \\
q(i)^{\eta}\Big(\gamma_L(i,h_L)l(i)+\gamma_H(i,h_H)h(i)\Big)^{1-\eta} ,&\quad \text{$i$ is not automated}
\end{dcases}. 
\end{align*}
When the intermediate goods $y(i)$ are produced with machine, the productivity is normalized to 1. The productivity of the skilled worker $\gamma_H(i,h_H)$ and the unskilled worker $\gamma_L(i,h_L)$ are increasing in the task index $i$ and the worker's human capital level $h_H$ and $h_L$. 
The task producers solve the following problem:
\begin{align*}
\max \quad  p(i)&y(i)-Rk(i)-W_Ll(i)-W_Hh(i)-\psi q(i).
\end{align*}

\subsubsection*{Research Sector}
Some skilled workers work as scientists in the R\&D sector to develop new automation or innovation patents. An innovation patent introduces a new task to the economy and increases the productivity of labor, automation patent allows the firm to produce an existing task using machines and improves the production factor allocation. After developing new patents, scientists sell task-specific intermediates to task producers at the price $\psi$ until the patent is outdated and destroyed. When a patent is destroyed, the patent holder is forced to exit the market and gets compensated by the new entry.\footnote{I borrow this structure of intellectual property rights from Acemoglu and Restrepo(2018)\nocite{AcemogluRestrepo2018} to exclude the creative destruction. They show that the structure with creative destruction has similar implications but introduces additional unstability to the model.} The values of automation and innovation patents are defined as the sum of discounted profit by selling patent intermediates, denoted by $P_N$ and $P_I$, respectively. The amount of automation and innovation patents introduced to the economy ($\kappa_I$ and $\kappa_N$) are functions of scientists hired in each sector ($\epsilon_I$ and $\epsilon_N$).
Assume that the production functions of the R\&D sector take the form: 
\begin{align}
\label{kappa_I}
\kappa_I(\epsilon_I) &= \frac{\epsilon_I^\lambda}{\mu_I}, \\
\label{kappa_N}
\kappa_N(\epsilon_N) &= \frac{\epsilon_N^\lambda}{\mu_N}, \quad 0<\lambda\leq 1.
\end{align}

The amount of new patents developed is increasing in the scientists working in the sector, with decreasing marginal productivity. Bloom et al. (2020)\nocite{Bloometal2020} find empirical evidence showing that research effort is rising while research productivity is declining. $\lambda$ captures the decreasing marginal productivity. $\mu_I$ and $\mu_N$ capture the inverse of research efficiency. The research is more costly with higher $\mu_I$ and $\mu_N$. 

\subsection{Static Model Equilibrium}
In this subsection, I start with a static version of the model to characterize the short-run effects of automation and human capital on allocation, factor prices and labor share. In section 3, I bring the static framework to a dynamic environment, to understand the interaction between the change of technology and human capital in the long-run. 

By solving the final good producers' problem, I can derive the demand function for task $i$ as a function of the task price $p(i)$, the demand for task $i$ is decreasing in the task price $p(i)$:
\begin{align}
\label{demand}
y(i) = \tilde{A}^{\sigma-1}Yp(i)^{-\sigma}.
\end{align}

\noindent{\bf Proof.} See Appendix A.

The task producers sell the tasks to final good producers. The task sector is perfectly competitive, so the price of the task $p(i)$ equals the cost of production. The effective cost depends on the ratio between the factor price and the factor productivity of the task. When the task has been automated, the task producers choose between capital and labor to minimize the production cost; when the task is new and has not been automated, the task producers can only select between high-skill and low-skill labor. Thus, the price $p(i)$ of task $i$ can be written as: 
\begin{align*}
p(i) &= 
\begin{dcases}
\Psi \min\{R^{1-\eta},\Big(\frac{W_L}{\gamma_L(i,h_L)}\Big)^{1-\eta},\Big(\frac{W_H}{\gamma_L(i,h_H)}\Big)^{1-\eta}\} &\text{, if automated}  \\
\Psi \min\{\Big(\frac{W_L}{\gamma_L(i,h_L)}\Big)^{1-\eta},\Big(\frac{W_H}{\gamma_L(i,h_H)}\Big)^{1-\eta}\} &\text{, if new,} 
\end{dcases} \\
\text{where } \Psi &= (\frac{\psi}{\eta})^{\eta}(\frac{1}{1-\eta})^{1-\eta}.
\end{align*}

 
\noindent{\bf Proof.} See Appendix A.

By plugging the price $p(i)$ into the demand function (Equation \ref{demand}) derived from the final good sector above, the output of each task can be solved as: 
\begin{align*}
p(i)y(i) &= 
\begin{dcases}
Y\Big(\frac{R}{A}\Big)^{1-\hat{\sigma}} &\quad  \text{, using capital}  \\
Y\Big(\frac{W_L}{A\gamma_L(i,h_L)}\Big)^{1-\hat{\sigma}} &\quad  \text{, using low-skill labor}    \\
Y\Big(\frac{W_H}{A\gamma_H(i,h_H)}\Big)^{1-\hat{\sigma}} &\quad  \text{, using high-skill labor}, 
\end{dcases} \\
\text{where } \hat{\sigma} &= 1-(1-\eta)(1-\sigma)\text{ and } A = \Big(\frac{\tilde{A}}{\Psi}\Big)^{\frac{\sigma-1}{\hat{\sigma}-1}}  = \Big(\frac{\tilde{A}}{\Psi}\Big)^{\frac{1}{1-\eta}}.
\end{align*}


\noindent{\bf Proof.} See Appendix A.

\begin{assumption}{\bf (Elasticity of Substitution)} \\

Assume that $\hat{\sigma}>1$.
\end{assumption}

When $\hat{\sigma}>1$, the tasks are substitutes. The output of the task is increasing in the task productivity and decreasing in the factor price. The demands for factors are decreasing in the factor prices. More workers are hired when their task productivity is higher. 


\begin{assumption}{\bf (Task Productivity)} \\

Assume that skilled workers have a comparative advantage in high index tasks over machine and unskilled workers; all workers have an absolute advantage over machine. The task productivity is increasing in the index $i$ and human capital $h_H$ and $h_L$. The productivity of machine is constant over all tasks and is normalized to 1. 
\begin{align*}
\gamma_K &= 1 \\
\gamma_H(i,h_H) &=e^{B(i)} e^{bh_H} \\
\gamma_L(i,h_L) &= e^{B(N-1)+B_L(i-(N-1))} e^{bh_L}
\end{align*}
\end{assumption}

The innovation patent is complementary to skilled workers and directly increases their productivity at the rate $B$. The innovation also has a spillover effect on unskilled workers and increases their productivity.\footnote{To assure the existence of a balanced growth path. Without the spillover effect, the wage premium $W_H/W_L$ converges to infinity in the long run.} The productivity of skilled and unskilled workers is always the same for the most outdated task ($N-1$), but skilled workers have a comparative advantage over unskilled workers in new tasks ($B-B_L$). The newer the tasks, the more comparative advantage the skilled workers have. Task productivity is also increasing in the human capital level. Human capital is general and increases workers' productivity for all tasks.  


\begin{proposition}{\bf (Allocation)} \\

The research sector always automates from the lowest index tasks. There exists automation level $I$, such that all the tasks between $N-1$ and $I$ are produced by machine, and all the tasks between $I$ and $N$ are produced by labor. Given automation level $I$, there exists a cutoff point $S$, such that all the tasks between $I$ and $S$ are produced by low-skill labor, and all the tasks between $S$ and $N$ are produced by high-skill labor. 

The task producers have incentives to adopt the newest automation patent when the following condition is satisfied:
\begin{align*}
R < \frac{W_L}{\gamma_L(I,h_L)}
\end{align*}
The task producers have incentives to adopt the newest innovation patent when the following condition is satisfied: 
\begin{align*}
R > \frac{W_H}{\gamma_H(N,h_H)}
\end{align*}
The allocation $S$ must satisfy the no arbitrage condition: 
\begin{align*}
\frac{W_H}{\gamma_H(S,h_L)} = \frac{W_L}{\gamma_L(S,h_L)}
\end{align*}
\end{proposition}
\noindent{\bf Proof.} See Appendix B.

\begin{figure}[h!]
\center
\includegraphics[width=0.8\textwidth]{allocation}
\caption{Task Allocation}
\label{allocation}
\end{figure}
Figure \ref{allocation} presents the task allocation. Workers have a comparative advantage over machine on newer tasks. The R\&D sector always automates the oldest tasks that have not been automated because the automation patent can generate more profit in the more outdated tasks. The R\&D sector has incentives to develop the automation patent only if the firm is willing to purchase the automation patent once the patent is developed; so all the automated tasks are produced using machine. For the tasks that have not been automated, skilled workers have more advantage in the high index tasks, so skilled workers take the newest tasks, and the unskilled workers take the less updated tasks. 


\begin{proposition}{\bf (Equilibrium in Static Model)} \\

After solving the firm's optimization problem and applying market clearing condition, the aggregate production function can be written as a constant elasticity of substitution (CES) production function, where the productivity of each production factor is determined endogenously by allocation: 
\begin{align}
\label{output}
Y = \frac{A}{1-\eta}\Big((\Gamma_KK)^{\frac{\hat{\sigma}-1}{\hat{\sigma}}}+(\Gamma_LL_L)^{\frac{\hat{\sigma}-1}{\hat{\sigma}}}+(\Gamma_HL_H)^{\frac{\hat{\sigma}-1}{\hat{\sigma}}}\Big)^{\frac{\hat{\sigma}}{\hat{\sigma}-1}}.
\end{align}
The equilibrium price and factor share can be solved as in a standard CES production function: 
\begin{align*}
R &=A\Gamma_K\Big(\frac{(1-\eta)Y}{A\Gamma_KK}\Big)^{\frac{1}{\hat{\sigma}}},   \quad s_K= \frac{RK}{Y},  \\
W_H &=A\Gamma_H\Big(\frac{(1-\eta)Y}{A\Gamma_HL_H}\Big)^{\frac{1}{\hat{\sigma}}},  \quad s_H = \frac{W_HL_H}{Y},  \\
W_L &= A\Gamma_L\Big(\frac{(1-\eta)Y}{A\Gamma_LL_L}\Big)^{\frac{1}{\hat{\sigma}}},  \quad s_L = \frac{W_LL_L}{Y},   \\
\omega &= \frac{W_H}{W_L} = \Big(\frac{\Gamma_H}{\Gamma_L} \Big)^{\frac{\hat{\sigma}-1}{\hat{\sigma}}}\Big(\frac{L_L}{L_H} \Big)^{\frac{1}{\hat{\sigma}}}, 
\end{align*}
where $\Gamma_K$, $\Gamma_L$, and $\Gamma_H$ represent the productivity of capital, low-skill and high-skill labor and are determined endogenously by allocation: 
\begin{align}
\label{Gamma_K}
\Gamma_K(\tilde{I}) &= \tilde{I}^{\frac{1}{\hat{\sigma}-1}}, \\
\label{Gamma_H}
\begin{split}
\Gamma_H(N,\tilde{I},\tilde{S},h_H)  &= \gamma_H(N,h_H)\Big(\frac{1-e^{-B(1-\tilde{S})(\hat{\sigma}-1)}}{B(\hat{\sigma}-1)}\Big)^{\frac{1}{\hat{\sigma}-1}} \\
&= \underbrace{\gamma_H(N,h_H)}_{\text{Tech frontier}}\underbrace{\tilde{\Gamma}_H(\tilde{S})}_{\text{Allocation}},
\end{split}
\\
\label{Gamma_L}
\begin{split}
\Gamma_L(N,\tilde{I},\tilde{S},h_L)  &= \gamma_L(N,h_L)\Big(\frac{e^{-B_L(1-\tilde{S})(\hat{\sigma}-1)}-e^{-B_L(1-\tilde{I})(\hat{\sigma}-1)}}{B_L(\hat{\sigma}-1)}\Big)^{\frac{1}{\hat{\sigma}-1}}\\
 &=\frac{\gamma_H(N,h_H)}{\gamma_{HL}}\tilde{\Gamma}_L(\tilde{I},\tilde{S}).
 \end{split}
\end{align}
$\tilde{I} = I-(N-1)$ represents the automation level, and $\tilde{S} = S-(N-1)$ represents the labor allocation. $\gamma_{HL} = e^{(B-B_L)+bh_{HL}}$ represents the absolute advantage of skilled labor over unskilled labor on the frontier task $N$, $h_{HL} = h_H-h_L$ represents the human capital gap between skilled and unskilled workers.  
\end{proposition}

As shown in Equation (\ref{Gamma_K}), capital productivity $\Gamma_K$ is a function of automation. When more tasks are automated, task producers use the machine to replace labor, $\Gamma_K$ increases, but $\Gamma_L$ and $\Gamma_H$ decrease. An increase in automation level increases the demand for capital but reduces the demand for both types of labor. As a result, the capital share increases, and the labor share decreases. Innovation does the opposite; it introduces new labor tasks, destroys old machine tasks, and increases labor productivity. 

As shown in Equation (\ref{Gamma_H}) and (\ref{Gamma_L}), labor productivity $\Gamma_H$ and $\Gamma_L$ depend on the allocation and human capital level. Human capital accumulation increases $\Gamma_L$ and $\Gamma_H$, which raises the wage level and labor share. $\Gamma_H$ and $\Gamma_L$ can be decomposed into two parts. The first part, $\gamma_H(N,h_H)$ or $\gamma_L(N,h_L)$, represents the technology frontier, the highest task productivity of each skill group. The second part, $\tilde{\Gamma}_H$ or $\tilde{\Gamma}_L$, captures the allocation. i.e., the measure of tasks produced by each type of worker. The ratio between $\Gamma_H$ and $\Gamma_L$ is increasing in the comparative advantage of skilled workers $B-B_L$ and the human capital gap $h_{HL}$. More tasks are produced by skilled workers when skilled workers have a higher comparative advantage and a larger human capital gap than unskilled workers. The skill wage premium increases when more tasks are assigned to skilled workers. 


\begin{proposition}{\bf (Comparative Statistics)} \\
The change in labor share is given by the following equation: 
\begin{align}
\label{ds_L}
\begin{split}
d\ln(s_{L}+s_{H}) &= \frac{\hat{\sigma}-1}{\hat{\sigma}}\frac{s_K}{1-\eta}\underbrace{( \frac{s_L}{s_L+s_H}\frac{d\ln\tilde{\Gamma}_L}{d\tilde{I}}-\frac{d\ln H}{d\tilde{I}})d\tilde{I}}_{\text{Automation}}\\
&-\frac{\hat{\sigma}-1}{\hat{\sigma}}\frac{s_K}{1-\eta}\underbrace{(d\ln K -BdN-bdh-d\ln L)}_{\text{Capital deepening}},
 \end{split}
\end{align}
where the average change of human capital and labor supply is defined as: 
\begin{align*}
d\ln h &= \frac{s_H}{s_H+s_L}d\ln h_H+\frac{s_L}{s_H+s_L}d\ln h_L,\\
d\ln L &= \frac{s_H}{s_H+s_L}d\ln L_L+\frac{s_L}{s_H+s_L}d\ln L_L.
\end{align*}
The change of allocation $\tilde{S}$ and wages can be written as: 
\begin{align}
\label{dS}
\underbrace{d\tilde{S}}_{\text{Allocation}} &= \frac{1}{\epsilon(\tilde{S})}\Big(\underbrace{\frac{1}{\hat{\sigma}}(d\ln L_L-d\ln L_H-bdh_{HL})}_{\text{Labor supply}}-\underbrace{\frac{\hat{\sigma}-1}{\hat{\sigma}}\frac{d\ln\tilde{\Gamma}_L}{d\tilde{I}}d\tilde{I}}_{\text{Automaton}}\Big),\\ 
\label{dW_L}
d \ln W_L &= \underbrace{d \ln Y}_{\text{Productivity}} + \underbrace{\frac{\hat{\sigma}-1}{\hat{\sigma}}\frac{d\ln\tilde{\Gamma}_L}{d\tilde{I}}d\tilde{I}}_{Displacement}+ \underbrace{\frac{\hat{\sigma}-1}{\hat{\sigma}}\frac{d\ln\tilde{\Gamma}_L}{d\tilde{S}}d\tilde{S}}_{\text{Relocate}},\\
\label{dW_H}
d \ln W_H &= \underbrace{d \ln Y}_{\text{Productivity}}  + \underbrace{\frac{\hat{\sigma}-1}{\hat{\sigma}}\frac{d\ln\tilde{\Gamma}_H}{d\tilde{S}}d\tilde{S}}_{\text{Relocate}}, \\
\label{dw}
d \ln \omega &=(B-B_L)d\tilde{S}+bdh_{HL},
\end{align}
where $\epsilon(\tilde{S})$ is the inverse of allocation elasticity taking the form:  
\begin{align}
\label{eS}
\epsilon(\tilde{S})&= \underbrace{\frac{\hat{\sigma}-1}{\hat{\sigma}}(\frac{d\ln \tilde{\Gamma}_L}{d\tilde{S}}-\frac{d\ln \tilde{\Gamma}_H}{d\tilde{S}})}_{\text{Wage elasticity}}
								 +\underbrace{(B-B_L)}_{\text{Comparative advantage}}.
\end{align}
\end{proposition}

\begin{figure}[h!]
\includegraphics[width=\textwidth]{comparative}
\caption{Comparative Statistics}
\label{comparative}
{\scriptsize Notes: Blue lines show the accumulative output share in the benchmark model. Red lines in the left panel show the change after an increase in automation; red lines in the right panel show the change after an increase in the human capital of skilled workers.}
\end{figure}

The left panel of Figure \ref{comparative} shows how output distribution changes after an increase in automation, i.e., $I$ shifts to the right. According to Equation (\ref{ds_L}), an increase in automation always decreases the labor share since it replaces unskilled workers directly. As shown in Equation (\ref{dS}), right after an increase in automation, the wage of unskilled workers decreases through the displacement effect shown in Equation (\ref{dW_L}). At the original cutoff point $S$, the unit cost of using unskilled workers is lower than using skilled workers. The unskilled workers then relocate and take on some tasks that used to be performed by skilled workers until the no arbitrage condition is satisfied again. The shift of allocation $S$ depends on the allocation elasticity derived in Equation (\ref{eS}). The labor allocation change is decreasing in the wage elasticity and comparative advantage of skilled workers. When the wage is elastic to the allocation, the no arbitrage condition can be satisfied with a small allocation change. With a high comparative advantage of skilled workers, the allocation is sticky since it's difficult to use unskilled workers to substitute the skilled workers.

The responses of wage levels are calculated in Equation (\ref{dW_L}) and (\ref{dW_H}). Since automation improves the allocation, it increases the wage for all the workers through the productivity effect. However, it decreases the wage of unskilled workers directly through the displacement effect and the wage of skilled workers through relocation effect. After an increase in automation, the wage premium, as shown in Equation (\ref{dw}), always increases. Automation pushes unskilled workers to tasks where they have a disadvantage compared to skilled workers. The change in wage premium is increasing in the comparative advantage and decreasing in the wage elasticity. When the comparative advantage is higher, the unskilled workers absorb more impact from automation since relocation is difficult. Unskilled workers have to accept a much lower wage to compete with skilled workers because their productivity is much lower at those tasks. When the wage elasticity is higher, the unskilled workers can pass more effects of automation to skilled workers since they can benefit more from the relocation.

The right panel of Figure \ref{comparative} shows how output distribution changes after an increase in the human capital of skilled workers, i.e., an increase in $h_H$. Automation and human capital are two opposing forces which together determine the labor share. According to Equation (\ref{ds_L}), human capital accumulation increases the labor share by increasing the effective labor supply. The total change in the wage premium after an increase in automation also depends on the human capital responses. An increase in $h_H$ increases the productivity of skilled workers, and then it's harder for unskilled workers to take over their tasks, the allocation $S$ shifts to the left. As shown in Equation (\ref{dw}), if the human capital gap is enlarged after an increase in automation, the effect of automation on inequality will be amplified. 

Automation and human capital are racing each other while jointly determine the labor share. Both automation and human capital gap can drive up the college wage premium. In the next section, I move this static setting into a dynamic environment to understand the interaction between human capital and automation so that I can analyze the full effect of automation. 

\section{Dynamics and Balanced Growth Path}
In this section, I extend the static framework above into a dynamic environment. First, capital accumulation is endogenous. Capital accumulation responds to the change in technology and human capital growth. Second, human capital growth is endogenous, depending on the time spent on training. Third, the R\&D is endogenous and directed. The automation level and technology growth rate result from R\&D investment.

\subsection{Balanced Growth Path}
\subsubsection*{Household}
Given the interest rate $r(t)$, wage $W_H(h_H,t)$ and $W_L(h_L,t)$, technology growth rate $g_N(t)$ and $g_I(t)$, and human capital growth rate $g_{hH}(t)$ and  $g_{hL}(t)$, households maximize their lifetime utility. By solving the households' problem, I can solve the intertemporal Euler equation, which characterizes the consumption-saving decision: 
\begin{align*}
\frac{\dot{C}_H}{C_H} &= \frac{\dot{C}_L}{C_L}  = \frac{r-\rho}{\theta}.
\end{align*}

The intratemporal Euler equation characterizes the optimal training decision for unskilled and skilled workers $j\in\{L,H\}$:
\begin{align*}
&\underbrace{\frac{\delta \log W_j(h_j)l_j}{\delta h_j}\frac{\delta \dot{h}_j}{\delta (1-l_j)}}_{\text{Direct wage gain}}+\underbrace{\frac{\delta \log W_j(h_j)}{\delta h_j}\dot{h}_j+\frac{\delta \log W_j(h_j)}{\delta t}}_{\text{Return to human capital}}= r(1+\tau_{hj}).
\end{align*}
Physical and human capital are two ways for workers to save for the future. The training incentive is decreasing in interest rate $r$ since a high interest rate increases the opportunity cost of training. The training incentive is increasing in return to human capital. Higher wage growth makes the household more willing to put effort into training.

By plugging in the production function, the marginal return rate of training is constant $b$, since the human capital is general and the growth rate of wage is linear in the human capital increase: 
\begin{align*}
\frac{d\log W_H(h_H,t)}{dh_H} = \frac{d\log W_L(h_L,t)}{dh_L} = b.
\end{align*}
Equations (\ref{gW_H}) and (\ref{gW_L}) characterize the wage growth rates of skilled and unskilled workers. They depend mainly on technology and human capital growth rate. The growth rate of physical capital and labor supply also has an impact on the wage growth rate:
\begin{align}
\label{gW_H}
\frac{dW_H(h_H,t)}{dt} &= g_{WN}+g_{WI}+\frac{s_L}{s_H+s_L}g_{WS},\\
\label{gW_L}
\frac{dW_L(h_L,t)}{dt}  &= g_{WN}+g_{WI}-\frac{s_H}{s_H+s_L}g_{WS}.
\end{align}
I decompose the wage growth into three components. $g_{WN}$ and $g_{WI}$ are common effects of innovation and automation on wage growth for both skilled and unskilled workers. The difference between the skilled and unskilled workers' wage growth is contributed mainly by the shift of allocation $g_{WS}$:  
\begin{align}
\label{g_WN}
g_{WN} &= \underbrace{Bg_N}_{\text{Productivity}}+\underbrace{a(\tilde{I})(g_K-g_L-Bg_N-bg_h)}_{\text{Capital deepening}}, \\
\label{g_WI}
g_{WI} &= \underbrace{a(\tilde{I})\frac{d\ln H}{d\tilde{I}}(g_I-g_N)}_{\text{Productivity}}+\underbrace{\frac{s_L}{s_H+s_L}(1-a(\tilde{I}))\frac{d\ln \tilde{\Gamma}_L}{d\tilde{I}}(g_I-g_N)}_{\text{Displacement}}, \\
\label{g_WS}
g_{WS}&= \frac{B-B_L}{\epsilon(\tilde{S})}(\underbrace{\frac{1}{\hat{\sigma}}(g_{LL}-g_{LH}+b(g_{hL}-g_{hH}))}_{\text{Labor supply}}-\underbrace{\frac{\hat{\sigma}-1}{\hat{\sigma}}\frac{d\ln \tilde{\Gamma}_L}{d\tilde{I}}(g_I-g_N)}_{\text{Relocation}}),
\end{align}
where $a(\tilde{I})$ is defined as the importance of capital:
\begin{align*}
a(\tilde{I}) &= \frac{1}{\hat{\sigma}}\frac{s_K}{1-\eta},
\end{align*}
$g_{LH}$ and $g_{LL}$ are growth rate of labor supply, and $g_K$ is growth rate of capital supply:
\begin{align*}
g_h &= \frac{s_H}{s_H+s_L}g_{hH}+\frac{s_L}{s_H+s_L}g_{hH}, \\
g_L &= \frac{s_H}{s_H+s_L}g_{LH}+\frac{s_L}{s_H+s_L}g_{LH}.
\end{align*}

$g_{WN}$ (Equation \ref{g_WN}) is composed of two standard growth components: increase of technology frontier and capital accumulation. An increase in innovation always increases the wage for all the workers since it pushes the technology frontier and increases their productivity. $g_{WI}$ (Equation \ref{g_WI}) is the common effect of automation on all the skill groups. Automation improves allocation and increases productivity but decreases wage growth through the displacement effect. $g_{WS}$ (Equation \ref{g_WS}) captures the uneven wage growth as a result of the change in allocation. Automation decreases unskilled workers' wages more than skilled workers. The human capital investment of different skill groups is complementary. Skilled workers have more incentive to invest in human capital if unskilled workers increase their time spent on training and vice versa. 

\subsubsection*{R\&D Sector}
After developing new tasks, the R\&D sector produces task-specific intermediates and sells them to task producers. The production function of task producers is assumed to be Cobb-Douglas, the patent profit is a constant share $\eta$ of the output: 
\begin{align*}
\text{Automation: } \pi_I(t)& =\eta y(i) = \eta(\frac{R(t)}{A})^{1-\hat{\sigma}}Y(t), \\
\text{Innovation: } \pi_N(i,t)&= \eta y(i) = 
\begin{dcases}
\eta(\frac{W_H(t)}{A\gamma_H(i,h_H(t))})^{1-\hat{\sigma}}Y(t),  \quad & i>S(t)  \\
\eta(\frac{W_L(t)}{A\gamma_L(i,h_L(t))})^{1-\hat{\sigma}}Y(t),  \quad & i \leq S(t).
\end{dcases}
\end{align*}

Similar to the task output, the patent profit is decreasing in the factor price but increasing in the task productivity. The increase in automation increases capital demand and decreases labor demand. With a higher capital price and lower wage level, the automation patent value rises, and the innovation patent value falls. The development of innovation patents does the opposite. 

When $\hat{\sigma}>1$, accumulating human capital increases task productivity more than the wage level, increasing the output and patent profit. Higher innovation patent value increases the incentive of the R\&D sector to innovate new tasks. The human capital and innovation R\&D investment are complementary. 

The patent value can be derived as the net gain of the new patent. When new task producers enter the market, they need to compensate the incumbent firms, who are forced to exit the market. Government imposes taxes on patent profit, $\tau_I$ for automation patents, and $\tau_N$ for innovation patents. The value of automation patent $P_I$ and innovation patent $P_N$ are derived in the following equations: 
\begin{align*}
P_N(t) &= (1-\tau_N)V_N(N(t),t)-(1-\tau_I)V_I(t), \\
P_I(t) &= (1-\tau_I)V_I(t)-(1-\tau_N)V_N(I(t),t). 
\end{align*}
The value of innovation patent $P_N$ (automation patent $P_I$) is the total future profit generated by the innovation patent of task $N(t)$ (the automation patent of task $I(t)$), subtracted by the compensation paid to the incumbent—automation patent holder of task $N(t)-1$ (innovation patent holder of task $I(t)$). $V_I(t)$ and $V_N(i,t)$ are the discounted value of future profit by holding an automation patent and an innovation patent for task $i$ at time $t$ if the tasks are never destroyed: 
\begin{align*}
V_N(i,t) &= \int_{t}^{\infty} e^{-\int_{t}^{\tau}r(s)ds}\pi_N(i,\tau)d\tau, \\
V_I(t)&= \int_{t}^{\infty} e^{-\int_{t}^{\tau}r(s)ds}\pi_I(\tau)d\tau. 
\end{align*} 

Automated tasks are always produced by capital, and unskilled workers always produce the most outdated innovation task. The profitability of automation patents stays constant since the task productivity of machine is constant and normalized to 1. The profitability of innovation patents is decreasing over time because the comparative productivity of the task is decreasing as it's further to the technology frontier. It's straightforward to solve the expected profit of automation patent $V_I$ and the most outdated innovation patent $V_N(I(t),t)$ using the following equations: 
 \begin{align*}
&V_I(t)= \eta\int_t^{\infty} e^{-\int_{t}^{\tau}r(s)ds}(\frac{R}{A})^{1-\hat{\sigma}}Y(\tau)d\tau, \\
\begin{split}
&V_N(I(t),t) =\eta \int_t^{\infty} e^{-\int_{t}^{\tau}r(s)ds}(\frac{W_L(\tau)}{A\gamma(I(t),h_L(\tau))})^{1-\hat{\sigma}}Y(\tau)d\tau \\
&= \eta e^{(\hat{\sigma}-1)B_L\tilde{I}(t)}\int_t^{\infty} e^{-\int_{t}^{\tau}(r(s)+(\hat{\sigma}-1)B_Lg_N(s))ds}(\frac{W_L(\tau)}{A\gamma(N(\tau),h_L(\tau))})^{1-\hat{\sigma}}Y(\tau)d\tau.
\end{split}				
 \end{align*}
 
The newest tasks introduced by innovation patent $V_N(N(t),t)$ are produced firstly by skilled workers, then by unskilled workers when it becomes more outdated. To solve the value of the newest innovation patent, I define the value of the newest patent $N(t)$ at time $t$ be $V_{NH}(t)$ if skilled workers produce the task and $V_{NL}(t)$ if unskilled workers produce it. The values of $V_{NH}(t)$ and $V_{NL}(t)$ are derived as: 
\begin{align*}
\begin{split}
V_{NH}(t) &= \eta\int_t^{\infty} e^{-\int_{t}^{\tau}r(s)ds}(\frac{W_H(\tau)}{A\gamma(N(t),h_H(\tau))})^{1-\hat{\sigma}}Y(\tau)d\tau \\
								&= \eta\int_t^{\infty} e^{-\int_{t}^{\tau}(r(s)+(\hat{\sigma}-1)Bg_N)ds}(\frac{W_H(\tau)}{A\gamma(N(\tau),h_H(\tau))})^{1-\hat{\sigma}}Y(\tau)d\tau,
\end{split} \\
\begin{split}
V_{NL}(t) &=\eta \int_t^{\infty} e^{-\int_{t}^{\tau}r(s)ds}(\frac{W_L(\tau)}{A\gamma(N(t),h_L(\tau))})^{1-\hat{\sigma}}Y(\tau)d\tau \\
								&= \eta \int_t^{\infty} e^{-\int_{t}^{\tau}(r(s)+(\hat{\sigma}-1)B_Lg_N)ds}(\frac{W_L(\tau)}{A\gamma(N(\tau),h_L(\tau))})^{1-\hat{\sigma}}Y(\tau)d\tau.
\end{split}
\end{align*}
I define $\tilde{\tau}(t)$ as the stopping time when the task $N(t)$ is switched from skilled workers to unskilled worker, i.e., $S(\tilde{\tau}(t)) = N(t)$. The expected profit of innovation patent can be solved explicitly: 
\begin{align*}
\begin{split}
 V_N(N(t),t) = V_{NH}(t)-e^{-\int_{t}^{\tilde{\tau}(t)}(r(s)+(\hat{\sigma}-1)Bg_N))ds}V_{NH}(\tilde{\tau}(t)) \\
 +e^{-\int_{t}^{\tilde{\tau}(t)}(r(s)+(\hat{\sigma}-1)B_Lg_N))ds}V_{NL}(\tilde{\tau}(t)).
 \end{split}
\end{align*}

The no arbitrage condition needs to be satisfied between the production and R\&D sectors. The wage payment of working in the production sector must equal the payoff of working as a scientist. The number of scientists working in automation and innovation sector must satisfy:
\begin{align}
\label{epsilon_NI}
\frac{\lambda\epsilon_I(t)^{\lambda-1}}{\mu_I}P_I(t) = \frac{\lambda\epsilon_N(t)^{\lambda-1}}{\mu_N}P_N(t) = W_H(t).
\end{align}

\subsubsection*{Market Clearing and Consistency}
The human capital and technology growth rates chosen by the households and R\&D sector need to be consistent with the given aggregate growth rate:
\begin{align*}
g_{hH} = \frac{(1-l_H)^{\alpha_H}}{\mu_{hH}}, \quad g_{hL} = \frac{(1-l_L)^{\alpha_L}}{\mu_{hL}}, \quad g_N = \frac{\epsilon_N^{\lambda}}{\mu_N}, \quad g_I = \frac{\epsilon_I^{\lambda}}{\mu_I}.
\end{align*}
The capital and labor markets need to be clear. The total capital demand must equal the capital supplied by skilled and unskilled workers. The low-skill labor demand needs to equal the total working time of unskilled workers. The skilled workers divide their working time into high-skill labor supply and research labor in automation and innovation sector. The market clearing conditions are listed in the following equations: 
\begin{align*}
\int_{N-1}^N k(i) di &= K = \epsilon_H K_H + \epsilon_L K_L, \\\
\int_{N-1}^N l(i) di &= L_L = \epsilon_L l_L,\\
\int_{N-1}^N h(i) di &= L_H = \epsilon_H l_H-\epsilon_N-\epsilon_I.
\end{align*}

\subsubsection*{Balanced Growth Path}
After solving the dynamic problem of the households and R\&D sector, I can characterize the Balanced Growth Path (BGP). Define normalized variables $x$ as the original variables $X$ normalized by economic growth $e^{-\int_0^{t}g(\tau)d\tau}$ or technology frontier $\gamma_H(N,h_H)$. The economy's growth rate is defined as the total growth of technology and human capital $g = Bg_N+bg_{hH}$. A balanced growth path (BGP) is defined as a transition path on which the automation level and the labor allocation $\{\tilde{I},\tilde{S}\}$, the factor shares $\{s_K,s_L, s_H\}$, the rental rate of capital $\{r\}$, the labor supply $\{L_H, L_L\}$ and the growth rates $\{g_N, g_I, g_{hH},g_{hL}\}$ are constant over time. The normalized factor productivity $\{\tilde{\Gamma}_H, \tilde{\Gamma}_L, \gamma_{HL}\}$, capital, output and consumption $\{k, y, c_H, c_L\}$, labor wages $\{\omega_H, \omega_L\}$, expected patent profit $\{v_N,v_I,v_{NH},v_{NL}\}$ and patent value $\{p_N, p_I\}$ are also constant on BGP.

The Euler equation with normalized variables can be written as: 
\begin{align*}
\frac{\dot{c_H}}{c_H} = \frac{\dot{c_L}}{c_L} = \frac{r-\rho}{\theta}-g = 0.
\end{align*}
On the balanced growth path, the normalized consumption is constant, so the long run interest rate needs to satisfy: 
\begin{align}
\label{LRR}
r = \rho+\theta g = \rho+\theta(Bg_N+bg_{hH}).
\end{align}

With constant growth rate, interest rate and wages, the value of automation and innovation patents are constant and can be written as: 
\begin{align}
p_N &= (1-\tau_N)v_N(N)-(1-\tau_I)v_I,\\
p_I &= (1-\tau_I)v_I-(1-\tau_N)v_N(I), 
\end{align}
where $\tilde{t}-t =\frac{1-\tilde{S}}{g_N}$, and $v_N(N)$, $v_N(I)$ and $v_I$ can be solved explicitly: 
\begin{align}
\label{V_NN1}
\begin{split}
v_N(N) &= v_{NH}+e^{-(r-g+(\hat{\sigma}-1)B_Lg_N)(\tilde{t}-t)}v_{NL} \\
&\quad \quad \quad -e^{-(r-g+(\hat{\sigma}-1)Bg_N)(\tilde{t}-t)}v_{NH},
\end{split} \\
\label{V_NI1} 
v_N(I) &=e^{-(\hat{\sigma}-1)B_L(1-\tilde{I})}v_{NL}, \\
\label{V_I1}
v_I &= \eta\frac{1}{r-g}(\frac{R}{A})^{1-\hat{\sigma}}y.
\end{align}
$v_{NH}$ and $v_{NL}$ are also constant on the BGP, and can be written as: 
\begin{align}
\label{V_NH1} 
v_{NH} &= \eta\frac{1}{r-g+(\hat{\sigma}-1)Bg_N}(\frac{\omega_H}{A})^{1-\hat{\sigma}}y, \\
\label{V_NL1} 
v_{NL} &=\eta \frac{1}{r-g+(\hat{\sigma}-1)B_Lg_N}(\frac{\gamma_{HL}\omega_L}{A})^{1-\hat{\sigma}}y.
\end{align}
The value of patents depends mainly on the automation level. Suppose the long-run interest rate is not too low. In that case, I can see from Equation (\ref{V_I1}), (\ref{V_NH1}), and (\ref{V_NL1}) that an increase in automation decreases the wage level, decreases the patent value of automation, and increases the patent value of innovation.\footnote{See Acemoglu and Restrepo (2018)\nocite{AcemogluRestrepo2018} for more details.} Also, with a higher automation level, as shown by equation (\ref{V_NI1}), unskilled workers are performing the tasks they have more comparative advantages over the machine, lowering further the value of the automation patent. Higher technology growth rate $g_N$ has an ambiguous effect on the patent value. It decreases the profit of automation patents by driving up the long-run interest rate (Equation \ref{LRR}). At the same time, it decreases the profit of innovation patents by accelerating its depreciation. A higher human capital growth rate always decreases the profit of automation patents by increasing the interest rate. 

The automation level is constant on the BGP, and needs to converge to the level where the innovation and automation grow at the same rate: 
\begin{align}
\label{g_NI}
g_N = g_I. 
\end{align}
By plugging in the R\&D sector production function (Equation \ref{kappa_I} and \ref{kappa_N}) and no arbitrage condition between production and research sector (Equation \ref{epsilon_NI} and \ref{g_NI}), the value of automation and innovation patent needs to satisfy:
\begin{align}
\label{p_NI}
\frac{p_N}{p_I} = (\frac{\mu_N}{\mu_I})^{\frac{1}{\lambda}}.
\end{align}
\begin{proposition}{\bf (Human Capital Growth and Automation Level)} \\

The automation level on the BGP is decreasing in the human capital growth rate. There exists a human capital growth rate $g_{hH}^*$, such that the economy will never be fully automated if the human capital growth rate $g_{hH}>g_{hH}^*$. 
\end{proposition}
\noindent{\bf Proof.} See Appendix B.

Human capital complements innovation patents but substitutes automation patents. A higher human capital growth rate slows the capital deepening and drives up the long-run capital rental rate (Equation \ref{LRR}). With a higher interest rate, the automaton level is lowered to satisfy the equilibrium condition (Equation \ref{p_NI}).

In the long run, $g_N = g_I$, the allocation between working and training for unskilled and skilled workers $j\in\{L,H\}$ needs to satisfy:
\begin{align*}
\frac{b}{\mu_{hj}}(1-(1-\alpha_j)l_j)(1-l_j)^{\alpha_j-1}+Bg_N&= r(1+\tau_{hj}) 
\end{align*}
The time spent on training is decreasing in the interest rate and increasing in the technology growth rate. The interaction between human capital and technology rate is shown in equation (Equation \ref{g_hg_N}) by plugging in the long-run interest rate (Equation \ref{LRR}):
\begin{align} 
\label{g_hg_N}
(1-\theta)\frac{b}{\mu_hj}(1-l_j)^{\alpha_j}+\alpha\frac{b}{\mu_hj}l_j(1-l_j)^{\alpha_j-1} = \rho+(\theta-1)Bg_N.
\end{align}

\begin{proposition}{\bf (Human Capital and Technology Growth Rate)}

In the long run, if the intertemporal elasticity of substitution $\theta>1$, the human capital and technology growth are substitutes; if $\theta<1$, the human capital and technology growth are complements.
\end{proposition}
\noindent{\bf Proof.} See Appendix B.

When the intertemporal elasticity of substitution $\theta$ is greater than 1, the capital stock response to the technology change is small. The long-run interest rate increases more than the economic growth rate, and the return to physical capital dominates the return to human capital. As a result, workers prefer saving through physical capital to human capital, working more and training less. On the contrary, when $\theta$ is smaller than 1, the benefit of human capital dominates, and the workers spend more time on training. 

The human capital growth needs to converge to the same growth rate: 
\begin{align}
\label{g_HL}
g_{hH} = g_{hL}.
\end{align}

\begin{assumption}{\bf (Cost of Training)}

In order to have balanced growth path exist, assume that the cost of training for skilled workers is increasing in the human capital gap between skilled and unskilled workers: 
\begin{align}
\label{mu_HL}
\frac{\mu_{hH}}{\mu_{hL}} = e^{\lambda_h(h_{HL}-h_{HL}^*)}.
\end{align}
\end{assumption}

This assumption is necessary to guarantee the existence of a balanced growth path. Without the assumption, equation (\ref{g_HL}) and (\ref{mu_HL}) can't be satisfied simultaneously. To maintain a higher human capital gap, skilled workers face a higher cost of training. College costs have increased by 169\% since 1980, while the college premium has increased only by 14\%.\footnote{Source: CNBC}


\subsection{Long-Run Comparative Statics}
In this subsection, I discuss the long-run implications of technological revolution, modeled as a permanent decline in $\mu_I$. The technological revolution increases the R\&D sector efficiency; the automation tasks are now easier to develop. In this subsection, I focus on the case where human capital and technology growth are complementary, i.e., $\theta<1$. 

\begin{proposition}{\bf (Long-run Comparative Statics)} 
\begin{itemize}
\item A reduction in automation cost increases the automation level $\tilde{I}$.
\item A reduction in automation cost increases the technology growth rate $g_N$ and human capital growth rate $g_{hL}$ and $g_{hH}$. Human capital accumulation slows down the capital deepening and pushes back the automation level. 
\item  A reduction in automation cost increases the human capital gap $h_HL$ and amplifies wage inequality. The change in human capital gap depends on the human capital accumulation parameters $\alpha_H$ and $\alpha_L$ and the elasticity of human capital investment cost to human capital gap $\lambda_h$:
\begin{align*}
dh_{HL} = \frac{1}{\lambda_h}\frac{(\alpha_H-\alpha_L)l_L}{(1-l_L)\alpha_H+l_L\alpha_L}d\log g_h
\end{align*}
\end{itemize}
\end{proposition}
\noindent{\bf Proof.} See Appendix B.

A reduction in automation cost immediately increases the automation level. Since automation patents are now easier to develop, the measure of automated tasks increases. A higher automation level improves the allocation and increases productivity. The value of innovation patents increases as the automation level increases. The innovation rate thus increases. A higher innovation rate increases future wage growth and incentivizes more human capital investment. Higher human capital growth complements innovation but substitutes automation. Human capital accumulation increases the effective labor supply and slows capital deepening, favoring innovation and decreasing the long-run automation level. 

Although both skill groups respond to the technological revolution, their responses are uneven. Both skilled and unskilled workers benefit from the technological revolution's productivity effect, but unskilled workers are affected more by the displacement effect. As a result, human capital investment incentives are higher for skilled workers than unskilled workers. Also, skilled workers have a higher ability to adjust their human capital investment. Skilled workers invest more in human capital than unskilled workers until the human capital gap converges to a higher level. 

To summarize the main result of this section, a reduction in automation cost increases the automation level and complements the innovation sector. Human capital accumulation complements innovation but substitutes automation. It lowers the automation level and increases the labor share. The increase in the human capital gap amplifies the effect of technological change on wage inequality. 

\section{Social Planner's Problem }
In the next section, I solve the transition path computationally and implement counterfactual experiments. Before that, I first solve the social planner's problem (SPP) in this section to discuss the inefficiency of the decentralized economy (CE). The social planner's problem is subject to the same aggregate production function and laws of motion as in the decentralized economy. However, the social planner maximizes the total social welfare by allocating the final good between consumption and saving, allocating time between working and training, and allocating skilled workers between the production and R\&D sectors. 
\subsection{Solution of SPP}
The social planner maximizes social welfare by solving the following maximization problem:
\begin{align*}
&\rho V_{SPP}(K,N,I,h_H,h_L)  \\
&= \max_{C_H, C_L,L_H,L_L,\epsilon_N, \epsilon_I} \quad \{\epsilon_H u(C_H)+\epsilon_L u(C_L)+V_K\dot{K}+V_N\dot{N}+V_I\dot{I}+V_{h_H}\dot{h}_H+V_{h_L}\dot{h}_L\},
\end{align*}
subject to the resource constraint and laws of motion: 
\begin{align*}
\dot{K} &= Y-\delta K-\epsilon_H C_H-\epsilon_L C_L, \\
\dot{N} &= \frac{1}{\mu_N}\epsilon_N^{\lambda}, \\
\dot{I} &= \frac{1}{\mu_I}\epsilon_I^{\lambda}, \\
\dot{h}_H &= \frac{1}{\mu_{hH}} (1-\frac{L_H+\epsilon_N+\epsilon_I}{\epsilon_H})^{\alpha_H}, \\
\dot{h}_L &= \frac{1}{\mu_{hL}} (1-\frac{L_L}{\epsilon_L})^{\alpha_L}, \\
Y &= \frac{A}{1-\eta}\Big((\Gamma_KK)^{\frac{\hat{\sigma}-1}{\hat{\sigma}}}+(\Gamma_LL_L)^{\frac{\hat{\sigma}-1}{\hat{\sigma}}}+(\Gamma_HL_H)^{\frac{\hat{\sigma}-1}{\hat{\sigma}}}\Big)^{\frac{\hat{\sigma}}{\hat{\sigma}-1}}.
\end{align*}
Similar to the decentralized economy, the consumption-saving decision can be characterized by the intertemporal Euler equations:
\begin{align*}
\frac{\dot{C_H}}{C_H} = \frac{\dot{C_L}}{C_L} = \frac{Y_K-\delta-\rho}{\theta}.
\end{align*}
The total return to human capital investment equals the net return to physical capital. The total return to human capital includes the direct return to human capital and the growth of labor return. Thus, the human capital investment can be characterized by the intratemporal Euler equations: 
\begin{align*}
-\frac{\delta \dot{h}_H}{\delta L_H}\frac{Y_{h_H}}{Y_{L_H}}+\frac{\dot{Y}_{L_H}}{Y_{L_H}} &=\frac{L_H}{\epsilon_H}\frac{b \delta \dot{h}_H}{\delta(1-l_H)}+\frac{\dot{Y}_{L_H}}{Y_{L_H}} = \quad Y_K-\delta, \\
-\frac{\delta \dot{h}_L}{\delta L_L}\frac{Y_{h_L}}{Y_{L_L}}+\frac{\dot{Y}_{L_L}}{Y_{L_L}} &= \underbrace{\frac{L_L}{\epsilon_L}\frac{b \delta \dot{h}_L}{\delta(1-l_L)}}_{\substack{\text{Human capital}\\ \text{return}}}+\underbrace{\frac{\dot{Y}_{L_L}}{Y_{L_L}}}_{\text{Growth}} = \underbrace{Y_K-\delta}_{\substack{\text{Physical capital}\\ \text{return}}}. 
\end{align*}
Optimal investments in the R\&D sector need to satisfy the no arbitrage condition, i.e., the long-run productivity gain of R\&D investment equals the return to high-skill labor in the production sector: 
\begin{align*}
\frac{\frac{\delta \dot{I}}{\delta \epsilon_I}Y_I}{Y_K-\delta-\frac{\dot{Y}_{L_H}}{Y_{L_H}}} = \frac{\frac{\delta \dot{N}}{\delta \epsilon_N}Y_N}{Y_K-\delta-\frac{\dot{Y}_{L_H}}{Y_{L_H}}} = Y_{L_H}.
\end{align*}

\subsection{Externality}
To compare the balanced growth paths of SPP and CE, I use the prices in CE to substitute the derivatives in SPP.  
\begin{align*}
Y_K &= \frac{R}{1-\eta}, \quad Y_{L_H}= \frac{W_H}{1-\eta}, \quad Y_{L_L}= \frac{W_L}{1-\eta}, \\
\frac{Y_{N}}{Y} &=\frac{s_L}{1-\eta}(B-B_L)+\frac{1}{\hat{\sigma}-1}(\frac{\omega_H}{A})^{1-\hat{\sigma}}-\frac{1}{\hat{\sigma}-1}(\frac{R}{A})^{1-\hat{\sigma}}, \\
\frac{Y_{I}}{Y} &=\frac{1}{\hat{\sigma}-1}(\frac{R}{A})^{1-\hat{\sigma}}-\frac{1}{\hat{\sigma}-1}e^{B_L(1-\tilde{I})(1-\hat{\sigma})}(\frac{\omega_Lh_{HL}}{A})^{1-\hat{\sigma}}, \\
\frac{\dot{Y}_{L_H}}{Y_{L_H}} &= g_{WH} = g_{WN}+g_{WI}+\frac{s_L}{s_H+s_L}g_{WS} +bg_{hH},\\
\frac{\dot{Y}_{L_L}}{Y_{L_L}} &= g_{WL} = g_{WN}+g_{WI}-\frac{s_H}{s_H+s_L}g_{WS} +bg_{hL}.
\end{align*}
The return to capital ($Y_K$) and labor ($Y_{L_H}$ and $Y_{L_L}$) in SPP is higher than that in CE since only a fraction $\eta$ of the output in CE is paid to the patent holder. The total returns to automation ($Y_N$) and innovation ($Y_I$) in SPP are productivity gains by introducing additional automation and innovation patents to the economy. They are higher than the patent value in CE since patent holders in CE only get a fraction $\eta$ of the return to the new patents. The wage growth rate in SPP is consistent with CE. 

\begin{table}[h!]
\begin{center}
\scriptsize
\renewcommand{\arraystretch}{2}
\begin{tabular}{l|ll}
\hline \hline
\multirow{2}{*}{Euler equation} & SPP  &  
$\frac{\dot{C_H}}{C_H} = \frac{\dot{C_L}}{C_L} = \frac{\frac{R}{1-\eta}-\delta-\rho}{\theta}$ \\
& CE &$\frac{\dot{C_H}}{C_H} = \frac{\dot{C_L}}{C_L} = \frac{r-\rho}{\theta}$ \\\hline
\multirow{2}{*}{Innovation rate} & SPP  &  
$\frac{\lambda\epsilon_N^{\lambda-1}}{\mu_N}\frac{(1-\eta)P_N^{SPP}}{Y} = \frac{W_H}{Y} $  \\ 
& CE &$\frac{\lambda\epsilon_N^{\lambda-1}}{\mu_N}\frac{P_N}{Y}  = \frac{W_H}{Y}$ \\\hline
\multirow{2}{*}{Automation rate}  & SPP  &  
$ \frac{\lambda\epsilon_I^{\lambda-1}}{\mu_I}\frac{(1-\eta)P_I^{SPP}}{Y}  = \frac{W_H}{Y}$  \\
& CE &$ \frac{\lambda\epsilon_I^{\lambda-1}}{\mu_I}\frac{P_I}{Y}  = \frac{W_H}{Y}$ \\\hline
\multirow{2}{*}{Innovation patent} & SPP  &  
$\frac{P_N^{SPP}}{Y} = \frac{\frac{Y_N}{Y}}{\frac{R}{1-\eta}-\delta-g_{WH}}$  \\
& CE &$\frac{P_N}{Y} =(1-\tau_N)\frac{V_N(N)}{Y}-(1-\tau_I)\frac{V_I}{Y}$ \\\hline
\multirow{2}{*}{Automation patent} & SPP  &  
$\frac{P_I^{SPP}}{Y}= \frac{\frac{Y_I}{Y}}{\frac{R}{1-\eta}-\delta-g_{WH}}$  \\
& CE &$\frac{P_I}{Y}  =  (1-\tau_I)\frac{V_I}{Y}-(1-\tau_N)\frac{V_N(I)}{Y}$ \\\hline
\multirow{2}{*}{Skilled training} & SPP  &  
$\frac{L_H}{\epsilon_H}\frac{b\delta \dot{h}_H}{\delta(1-l_H)}+g_{WH} = \frac{R}{1-\eta}-\delta$  \\ 
& CE &$l_H\frac{b\delta \dot{h}_H}{\delta(1-l_H)}+g_{WH} =  r(1+\tau_{hH})$ \\\hline
\multirow{2}{*}{Unskilled training} & SPP  &  
$l_L\frac{b\delta \dot{h}_L}{\delta(1-l_L)}+g_{WL}= \frac{R}{1-\eta}-\delta$  \\ 
& CE &$l_L\frac{b\delta \dot{h}_L}{\delta(1-l_L)}+g_{WL}=  r(1+\tau_{hL})$ \\\hline
\end{tabular}
\end{center}
\renewcommand{\arraystretch}{1}
\caption{Comparison Between SPP and CE}
\label{SPP_CE}
\end{table}

Table \ref{SPP_CE} compares the equations characterizing the BGP of SPP and CE. First, the production externality of production and R\&D sectors is common in endogenous growth literature. The production sector does not internalize the patent profit, and the R\&D sector does not internalize the productivity gain of the production sector. The socially-planned economy always invests more capital and grows faster than the decentralized economy. I will focus on the second and third externalities in the following discussions.

Second, the skilled workers overinvest in human capital. When workers invest in human capital, they increase their productivity in the production sector. Comparing the skilled workers' human capital policy function of SPP and CE, I can find that skilled workers fail to internalize their negative effect on the R\&D sector. Higher labor productivity increases the opportunity cost of research and development; fewer scientists work in the R\&D sector. The number of scientists hired in the research sector is decreasing in the wage premium.

Third, the R\&D sector automates too many tasks. The innovation patent value of SPP is higher than that of CE. When an innovation task is developed, the patent holder only benefits from the productivity gain of the single task. However, the new task complements other tasks and increases the return to all the production factor providers and patent holders. At the same time, the new task pushes the technology frontier and increases the productivity of all the unskilled workers. 

Based on the social planner's problem, I will focus on an automation tax and a training tax on skilled workers in the counterfactual experiments of section 5. An automation tax decreases the value of automation patents and lowers the automation level; a training tax on skilled workers favors unskilled workers and reduces the human capital accumulation of skilled workers.

\section{Calibration and Transition Path}
I calibrate the model to match the moments in 1980 and 2005, and simulate the transition in between. Then I discuss the policy implications by conducting a variety of counterfactual experiments. 

\subsection{Calibration}
I use the data from 1980 and 2005 to calibrate the model parameters and the size of the automation cost reduction $z$. Acemoglu and Autor (2011)\nocite{AcemogluAutor2011} document that the structural change in the labor market started in 1982. We can observe a decline in labor share accompanied by an increasing college wage premium. Thus, I consider 1980 as the starting point of the technological revolution. I cut the data at 2005 to exclude the effect of the Great Recession. Restrepo (2015)\nocite{Restrepo2015} demonstrates an interaction between the Great Recession and the labor market structural change. All the parameters are assumed to be the same in 1980 and 2005 except for the automation cost. After the technology shock, the cost of automation decreases to $z\mu_I$; the economy gradually converges to a new balanced growth path. 

\begin{table}[h!]
\begin{center}
\scriptsize
\begin{tabular}{c|ld{1.1}|l}
\hline \hline
 \multicolumn{1}{c|}{Parameter} &\multicolumn{1}{l}{Description}     & \multicolumn{1}{c|}{Value}   & \multicolumn{1}{c}{Reference} \\ \hline 
$\theta$    & Intertemporal elasticity of substitution        &  0.9  & Beaudry and Van Wincoop (1996)  \\
$\sigma$    & Factor elasticity of substitution        &  2    \\
$\delta$    & Depreciation rate       &  0.1 & BEA Depreciation Estimates  \\
$B_L$ & Comparative advantage of unskilled workers   &     1  \\
$b$ & Return to human capital   &     1  \\
$\epsilon_H$    & high-skill workers share   &  0.3   & FRED \\
$\epsilon_L$     & Low skill workers share     &  0.7  & FRED \\
$\lambda$ & R\&D production function & 0.5 & Prettner and Strulik (2020)\\
\hline
\end{tabular}
\end{center}
\caption{External Calibration}
\label{calibration1}
\end{table}
The parameters in Table \ref{calibration1} are calibrated externally. Intertemporal elasticity of substitution takes the value of 0.9.\footnote{Beaudry and Van Wincoop (1996)\nocite{BeaudryVanWincoop1996} provide evidence that IES is significantly different from zero and close to one.} Factor elasticity of substitution is set to be 2. I set the rate of depreciation of traditional capital to $\delta = 0.1$ based on the depreciation rates suggested by the Bureau of Economic Analysis. The comparative advantage of unskilled workers and return to human capital are normalized to 1. $\epsilon_H$ and $\epsilon_L$ are set to match the share of workers with a college degree or higher and workers with a high school diploma or lower.\footnote{Data source: FRED} $\lambda$ in the R\&D sector production function follows the calibration from Prettner and Strulik (2020)\nocite{PrettnerStrulik2020}.

\begin{table}[h!]
\begin{center}
\scriptsize
\begin{tabular}{c|ld{3.4}|ld{1.3}}
\hline \hline
Parameter &  \multicolumn{1}{l}{Description}  & \multicolumn{1}{c|}{Value} &  \multicolumn{1}{l}{Targeted moment}  &  \multicolumn{1}{c}{Value}  \\ \hline
$\rho$    & Discount rate         &  0.0121    &  Long run interest rate & 4.0\%   \\
$\eta$    & Patent share                  & 0.1125     & RD and GDP ratio & 2.8\% \\
$A$       & Capital productivity               &     0.1190  & Short run interest rate      & 4.0\%   \\
$B_H$     & Skill comparative advantage  &  2.2651   & Wage premium (1980)      & 1.4    \\
$\mu_N$ & Cost of innovation   &   2.3619    & RD growth rate  & 2.8\%   \\
$\mu_I$ & Cost of automation   &   9.0135     & Labor share (1980)         & 0.625   \\
$\mu_h$ & Cost of $h_L$ accumulation   &   193.5608     & Human capital growth rate & 0.3\% \\
$\alpha_H$     & $h_H$ accumulation function   &   0.9026     & Change of training time  (H)  & 0.141 \\
$\alpha_L$     &  $h_L$ accumulation function   &   0.2797  & Change of training time  (L) & 0.160 \\
$\lambda$ & RD Decreasing return  &   	0.7675    & Wage premium (2005)   & 1.6 \\
$z$      & Technological revolution       &   0.7853  &  Labor share (2005)        & 0.605    \\\hline
\end{tabular}
\end{center}
\caption{Internal Calibration}
\label{calibration2}
{\scriptsize Notes: Data are from FRED and ATES. All the macroeconomic moments are from FRED. The changes in training time are computed using ATES. The details on how I choose the moments are presented in Appendix C.}
\end{table}

The remaining parameters in Table \ref{calibration2} are calibrated using the Simulated Method of Moments (SMM), which aims at finding a value $\theta^*$ maximizing the likelihood function. $\hat{m}_S(\theta)$ is the moment simulated by the model given $\theta$ as the parameter of the model; $\hat{m}_M$ is the moment observed in the data.
\begin{align*}
L(\theta) = -(\hat{m}_S(\theta)-\hat{m}_M)^T(\hat{m}_S(\theta)-\hat{m}_M)
\end{align*}

The share of profit paid to the patent holder $\eta$ is set to match the R\&D investment and GDP ratio. The return to capital is, on average, 4\% in the data, which pins down the capital productivity $A$. The comparative advantage of skilled workers $B_H$ is calibrated to match the skill premium in 1980. The cost of innovation $\mu_N$ is set to match the long-run growth rate. The labor share is decreasing in the automation level, and the cost of automation determines the automation level. Thus, the cost of automation $\mu_I$ is set to match the labor share in 1980. Human capital investment cost $\mu_h$ is closely related to the human capital growth rate. The parameters in the human capital accumulation function $\alpha_H$ and $\alpha_L$ are picked to match the training time change for college and high school graduates. The size of technology shock $z$ is closely related to the change in labor share; after a reduction in automation cost, more tasks are automated, and labor share decreases. The details on how I choose the moments are presented in Appendix C.

\begin{table}[h!]
\begin{center}
\scriptsize
\begin{tabular}{l|d{1.5}|d{1.5}d{1.5}}
\hline \hline
\multicolumn{1}{l|}{Moment} & \multicolumn{1}{c|}{1980} &  \multicolumn{2}{c}{2005} \\ 
&   &  \multicolumn{1}{c}{Fixed} & \multicolumn{1}{c}{Endogenous} \\ 
&  & \multicolumn{1}{c}{human capital} &  \multicolumn{1}{c}{human capital} \\ \hline
Automation level & 0.6818  & 0.7796 & 0.7623\\
Labor share & 0.6174  & 0.5906 & 0.5939\\
\quad Skilled labor share & 0.2814  & 0.2758 & 0.2840 \\
\quad Unskilled labor share & 0.3360  & 0.3148 & 0.3099 \\
Wage premium   & 1.4012  & 1.4454  & 1.5897\\
Technology growth rate  & 4.2036\%  & 5.5177\% & 5.2122\%\\
Human capital growth rate  & 0.2612\%  & 0.2612\% & 0.2743\%\\
Welfare inequality  & 1.0087  & 1.0902  & 1.1859\\ \hline
\end{tabular}
\end{center}
\caption{Calibration Result}
\label{result}
{\scriptsize Notes: The first column presents the key moments before the technological revolution. The second and third columns present the key moments after the technological revolution with fixed or endogenous human capital.}
\end{table}

The calibration result is presented in Table \ref{result}. A reduction in automation costs increases automation, lowers labor share, and raises wage inequality. R\&D investment is directed more to automation since the automation department is more efficient. As more tasks are automated, labor share decreases, and wage inequality increases. However, the higher automation level increases the allocation efficiency, which increases the profit of innovation patents and the innovation rate. A 20\% reduction of the automation cost increases the technology growth rate by about 1\%. 

I compare the two scenarios with and without human capital. A higher innovation rate increases wage growth and complements human capital investment. Workers accumulate human capital at a higher rate, increasing the effective labor supply, pushing back automation, and increasing the labor share. Human capital accumulation increases the labor share by about 0.33\% (0.82\% from skilled labor share and -0.48\% from unskilled labor share). Skilled and unskilled workers respond differently to the reduction in automation costs, which increases the human capital gap. Only 23.4\% of the wage premium increase is contributed directly by automation, and the increase in the human capital gap causes 76.6\% of the increase. 

The technology growth rate is lower in the scenario with endogenous human capital due to the higher opportunity cost of research. The skilled workers are more productive in the production sector, and fewer skilled workers become scientists and work in the R\&D sector. 

\subsection{Transition Path}

The transition path between two balanced growth path before and after political or technological shock is consist of state variables capital stock $\{k(t)\}$, automation level $\{\tilde{I}(t)\}$, absolute advantage $\{\gamma_{HL}(t)\}$; growth rates of the economy $\{g_N(t), g_I(t), g_{hH}(t),g_{hL(t)}\}$; price functions $\{r(t), \omega_H(t), \omega_L(t)\}$; policy functions $\{f_{Hk}(t), f_{Lk}(t), f_{Hc}(t), f_{Lc}(t)\}$ and $\{f_{Hl}(t), f_{Ll}(t)\}$ for households; $\{f_{K}(t), f_{H}(t), f_{L}(t)\}$ for firms; $\{f_{\epsilon_N}(t), f_{\epsilon_I}(t)\}$ for R\&D sector, and allocation $\{\tilde{S}(t)\}$, such that: 
\begin{enumerate}
\item Given the state variables, growth rates and price functions, household policy functions  $\{f_{Hk}(t), f_{Lk}(t), f_{Hc}(t), f_{Lc}(t)\}$ and $\{f_{Hl}(t), f_{Ll}(t)\}$ solve household's problem
\item Given the state variables, growth rates and price functions, firm policy functions $\{f_{K}(t), f_{LH}(t), f_{LL}(t)\}$ solve firm's problem
\item Given the state variables, growth rates and price functions, R\&D sector policy functions $\{f_{\epsilon_N}(t), f_{\epsilon_I}(t)\}$ solve  R\&D sector's problem
\item Consistency and market clearing conditions are satisfied. 
\end{enumerate}

\begin{figure}[h!]
\includegraphics[width=\textwidth]{Transition}
\caption{Transition Path}
\label{transition}
{\scriptsize Notes: The figure shows the simulation of the transition path after a reduction in automation cost with the calibrated model. Red lines show the transition with fixed human capital investment; blue lines show the transition with endogenous human capital investment. The derailed algorithm is provided in Appendix D. Oscillations are caused by computational errors.}
\end{figure}

Figure \ref{transition} plots the simulation result of the transition path with the calibrated model. The detailed algorithm is provided in Appendix D. The simulation with fixed human capital investment (scenario 1) is plotted with blue lines, and the simulation with endogenous human capital (scenario 2) is plotted with red lines. 

In scenario 1, the human capital investment is fixed. A reduction in automation costs immediately increases the automation rate and the automation level increases. The technological change increases the demand for capital but decreases the demand for skilled and unskilled workers. As a result, the interest rate increases, and wages fall. Higher interest rates and lower wages increase the return to innovation patents and decrease the return to automation patents. Thus, more jobs are created for workers and the automation process slows down. The innovation rate continues to rise while the automation rate starts to decrease; both rates converge to a new balanced growth path with higher growth rate around year 5. 

In scenario 2, the human capital investment is endogenous. At the beginning, the productivity effect of automation dominates the displacement effect, workers increase their human capital investment, spend more time on training and supply less labor. The lower labor supply lowers the return to physical capital, and turns the human capital investment even more attractive. The labor share drops more compared with scenario 1, and the automation level overshoots. Labor share starts to recover around year 10. The innovation rate doesn't start to catch up until the workers accumulate enough human capital. The human capital accumulation slows down, the labor share continues to increase and eventually reaches a level higher than in scenario 1. When the innovation rate catches up, skilled workers accumulate their human capital faster than unskilled workers, since they benefit more from the new job creations. The wage premium keeps increasing and converges to a much higher level than in scenario 1.

Endogenous human capital decreases the long-run automation level and increases the long-run labor share. However, it amplifies the effects of the technological revolution on the labor market outcomes. The labor share drops more in the short-run, and the wage premium increases more in the long run. 

\subsection{Policy Implications}
In section 4, I solve the social planner's problem and discuss the externalities of the decentralized economy. In the decentralized economy, the R\&D sector automates too many tasks, and skilled workers invest too much in human capital. Two taxes are appropriate to correct the externalities and increase the overall social welfare: an automation tax ($\tau_I$) and a training tax on skilled workers ($\tau_{hH}$). An automation tax decreases the value of automation patents and lowers the automation level; a training tax on skilled workers favors unskilled workers and decreases the human capital accumulation of skilled workers. 

\begin{figure}[h!]
\includegraphics[width = \textwidth]{Tax}
\caption{Tax Analysis}
\label{tax}
{\scriptsize Notes: Equal transfers to all the workers are financed by alternative tax rates of an automation tax (blue lines) or a training tax on skilled workers (red lines). The figure shows the predicted economic aggregates for the new balanced growth path (after the technological revolution) with the calibrated model. The x-axis is the alternative tax rate. }
\end{figure}

I address the quantitative implications with the calibrated model and consider a redistributive policy. I consider the case in which all the workers receive equal transfers, which are financed by either an automation tax or a training tax on skilled workers in a revenue-neutral way. Results are shown in Figure \ref{tax}. Blue and red lines plot the predicted economic aggregates for the new balanced growth path with different tax rates of an automation tax and a training tax on skilled workers. 

In the long run, an automaton tax decreases the automation level and increases the labor share; fewer tasks are automated due to a lower profit of automation patents. Lower automation level decreases the wage and welfare inequality. On the other hand, a training tax increases the automation level and decreases the labor share. It favors automation and suppresses human capital investment. However, the human capital growth rate is higher than the aissez-faire rate. When the tax reduces the human capital investment, it decreases the opportunity cost of R\&D and increases the technology growth rate. Higher technology growth rate increases the wage growth and encourages more human capital investment. A training tax reduces the wage premium and welfare inequality more efficiently than an automation tax. A small tax on training significantly decreases the human capital gap and favors unskilled workers. 

In the long run, consistent with the result of the social planner's problem in section 4, both automation tax and training tax on skilled workers improve the economy's efficiency and boost long-run growth. The automation tax corrects the positive externality of innovation on aggregate productivity. The training tax on skilled workers corrects the negative externality of training in the R\&D sector. However, the long-run implication does not tell the whole story; I next plot the transition path while imposing taxes.  

\begin{figure}[h!]
\includegraphics[width=\textwidth]{Transition3}
\caption{Adjustment Dynamics (automation or training tax)}
\label{transition3}
{\scriptsize Notes: Blue lines show the simulation of the transition path after a reduction of automation cost, red lines show the transition with a 1\% automation tax, and yellow lines show the transition with a 0.7\% training tax on skilled workers. Oscillations are caused by computational errors.}
\end{figure}

Figure \ref{transition3} plots the transition path with a 1\% automation tax (blue lines) or a 0.7\% training tax (red lines) on skilled workers, which renders the same fiscal revenue as the automation tax. The automation tax does not change much the transition dynamics. It uniformly decreases the automation level and increases the labor share. On the other hand, the training tax significantly reshapes the transition path. Skilled workers invest less in training and supply more high-skill labor at the early stage of the technological revolution, the innovation rate catches up with the automation rate earlier, and the labor share recovers faster after the technological revolution. 

\section{Empirical Implications}
I study the implications of my model in the context of exposure to automation affecting skill levels and training time. I start by documenting recent changes in skill levels and human capital investment. Then I estimate the effect of automation on the human capital response at the industry and occupation levels. Finally, I explore the heterogeneous responses of college graduates or higher (skilled workers) and high school graduates or lower (unskilled workers) to the technology change. 

At the industry level, I use the EU KLEMS (EU level analysis of capital, labor, energy, materials and service inputs) and use Total Factor Productivity (TFP) growth as an approximation of the automation process following Autor and Salomons (2018)\nocite{AutorSalomons2018}. After a reduction of automation costs, both automation and innovation rates increase, leading to an increase in TFP. I use labor composition (share of college graduates) in the EU KLEMS to approximate human capital. 

At the occupation level, I use the automation occupational exposure constructed by Webb (2019)\nocite{Webb2019}, which measures how likely the workers in each occupation can be replaced by machine. I use the measure of skill level and importance in O*NET (Occupational Information Network) to approximate the human capital requirement and stock. I also use individual survey data to assess the human capital responses. I use the training time in ATES (Adult Training and Education Survey) to measure the investment in human capital. NLSY97 (National Longitudinal Survey of Youth 1997 Waves) provides information on workers' working history, wage history, and training activities. 

\subsection{Changes in Skill Level and Human Capital Investment}
I document recent trends in skill levels and human capital investment. At the industry level, I focus on the change of labor composition (share of college graduates or higher); at the occupation level, I focus on the skill level change and the time spent on training. 
 
To explore the human capital change at the industry level, I use the EU KLEMS September 2017 release, an industry-level panel dataset covering all individual EU-28 countries and the United States from 1995 to 2015. The data set contains statistical data on the economic growth (value added, compensation to capital and labor, etc.) and labor market composition (compensation and employment share for high/medium/low educated workers), and analytical data on the growth factors (TFP, capital services, hours worked, labor composition, etc.). 

Between 1995 and 2015, the aggregate share of skilled workers increased. There are significant variations between industries; industries using more unskilled workers like agriculture and manufacturing experienced more labor composition change, while industries using more skilled workers like education and health experienced less (Table \ref{Industry_trend} in Appendix E). First, workers respond to the technological change by investing more in human capital. Second, as shown in Restrepo (2015)\nocite{Restrepo2015}, automation replaces unskilled workers, and unskilled workers relocate to industries that are less exposed to automation. In subsection 6.2, I estimate the effect of automation on labor composition while controlling for the relocation effect. 

To explore the skill change at the occupation level, I use the O*NET (Occupational Information Network) data. I merge all the history releases of O*NET and construct panel data. For each occupation, O*NET measures the importance and level of different skills and updates the data almost twice a year. The first official O*NET data set was released in 2003, and a major change was made in 2010. In other releases, only a subset of occupations was updated. There are two groups of skills: basic skills, including content and process; cross-functional skills, including complex problem solving, resource management, social, systems and technical. Each subgroup includes several detailed skill types, which describe the specifically developed capacities needed for the occupation. 

Between 2000 and 2020, the trends in level and importance are very similar (Figure \ref{trend} in Appendix E). The main skill growth happened between 2002 and 2010, and only minor changes were made after the major update in 2010. The level and importance of social skills have grown significantly faster than other skills, while only technical skills are lower and less important than before.  I use Webb (2019)\nocite{Webb2019} to measure the automation exposure of each occupation, and the ratio of workers with bachelor's degree to measure the skill level of workers at each occupation. The skill levels of all the skill groups increase significantly more when the occupation is more exposed to automation, especially process, social and complex problem solving skills. The skill levels of the occupation grow significantly faster as more skilled workers work in this occupation, except for technical skills (Figure \ref{LV_trend1} and \ref{LV_trend2} in Appendix E).

A direct measure of human capital investment can be constructed using ATES (Adult Training and Education Survey). The ATES collects data about adults ages 16 to 65 who are not enrolled in high school. I use the hours per week in ESL (English as a Second Language) and ABE (Adult Basic Education)/GED (General Equivalency Diploma) classes multiplied by the number of weeks spent to construct the total hours spent on nondegree credentials. ATES also asks about the total hours spent on work experience programs. Other relevant data ATES provides are hours per week worked for pay, worker's occupation (20 aggregate groups) if employed, sex, age and ethnicity. 

\begin{figure}[h!]
\includegraphics[width = \textwidth]{train}
\caption{Training Facing Different Automation Exposure}
{\scriptsize Notes: Data are from the ATES (Adult Training and Education Survey). I calculate the average training-working time ratio and the ratio of workers participating in any type of training for each occupation. The x-axis is the automation occupational exposure constructed by Webb (2019).}
\label{train}
\end{figure}

Figure \ref{train} plots the average training-working time ratio and the ratio of workers participating in any type of training and their automation exposure for each occupation. Both ratios are decreasing in automation exposure. Workers exposed more to automation invest less in their human capital. First, unskilled workers exposed to automation face a higher risk of being replaced by the machine and not eligible for work-related courses. Second, displaced workers relocate to new occupations that are less exposed to automation and are more likely to participate in workplace training. 

A direct measure of human capital investment can be constructed using wage level. The NLSY97 contains economic, sociological, and psychological information on 8,984 young individuals who were between 12 and 18 years of age in 1997. In particular, it includes information on the worker's history occupation, history wage, and training participation. Since my analysis regards the after-school human capital responses to automation, I restrict the sample to workers who are not enrolled in schools. For each worker, I track his working history and construct his lifetime automation exposure based on his occupations' automation exposure and his job length at each occupation. I divide workers into two groups: skilled if the worker has a bachelor's degree or higher, and unskilled otherwise. 

\begin{figure}[h!]
\includegraphics[width = \textwidth]{wage_trend}
\caption{Wage Growth Facing Different Automation Exposure}
{\scriptsize Notes: Data are from the NLSY97 (National Longitudinal Survey of Youth 1997 Waves). Workers are skilled if he has a bachelor's degree or higher, and unskilled otherwise. Workers have high (low) automation exposure is his lifetime automation exposure is above (below) median. The x-axis is the survey year, the y-axis is the cumulative log change in hourly wages.}
\label{wage_trend}
\end{figure}

Figure \ref{wage_trend} plots the cumulative log change in hourly wages for skilled and unskilled workers facing different automation lifetime exposure. Automation has a more significant negative impact on wage growth for unskilled workers than for skilled workers. First, automation has a stronger displacement effect on unskilled workers. Second, facing the same automation exposure, skilled workers have a higher ability to adapt to technological change. 

In this subsection, I document that the labor composition, skill level change, training investment and wage respond differently based on automation exposure. It seems contradictory that occupations with higher automation exposure experience higher skill growth rates, but workers spend less time on training when their occupation is more exposed to automation. First, when an occupation is exposed to automation, unskilled workers are more likely to be displaced, while skilled workers are more likely to work at this occupation to complement the new technology. The labor composition change increases the average skill level in this occupation. Second, the growth in skill level is mainly attributed to skilled workers. Skilled workers can acquire skills to adapt to the new technology environment more quickly than unskilled workers. Indeed, the skill growth rate is higher for occupations with a higher ratio of skilled workers. Third, skills can be acquired outside formal workplace training. I will explore the heterogeneous responses in more detail in the following subsection. 

\subsection{Effects of Automation on Human Capital Investment}
Based on the model, after a reduction of automation cost, both automation and innovation rates increase. In the long run, workers respond to the change in technology by investing more in their human capital. As a result, we should observe an overall increase in human capital or an improvement in labor composition (higher ratio of skilled workers). To measure human capital growth, I use two related variables: the change in the share of skilled workers (college graduates) and the growth rate attributed to labor composition change. 

To estimate the effect of automation on labor composition, I adopt the local projection models in Autor and Salomons (2018)\nocite{AutorSalomons2018}. They document well in their paper the impact of automation on employment, hours worked, wage bills, and labor share. I adopt their estimation method while the outcome of interest is the labor composition and the growth rate attributed to labor composition change. I estimate the equation: 
\begin{align}
\begin{split}
\ln Y_{i c, t+K}-\ln Y_{i c, t-1}&=\beta_{0}+\beta_{1} \Delta \ln TFP_{i, c \neq c(i), t-1}+\sum_{k=0}^{K} \beta_{2}^{k} \Delta \ln TFP_{i, c \neq c(i), k} \\
&+\beta_{3} \Delta \ln TFP_{i, c \neq c(i), t-2}+\beta_{4} \Delta \ln Y_{i c, t-2}+\alpha_{c}+\gamma_{t}+\epsilon_{ict},
\end{split}
\end{align}
where $\ln Y_{i c, t+K}-\ln Y_{i c, t-1}$ represents the log change in share of high educated labor in industry $i$ and country $c$, from year $t-1$ to year $t+K$. The impulse variable is the log change in other-country-industry TFP between years $t-2$ and $t-1$, $\Delta \ln TFP_{i, c \neq c(i), t-1}$. I estimate the effects while controlling for lagged values of both TFP growth $\Delta \ln TFP_{i, c \neq c(i), t-2}$ and of outcome variable growth $\Delta \ln Y_{ic, t-2}$, and subsequent TFP innovations occurring between $t=0$ and $t=K$. I also control the fixed effect of country $\alpha_{c}$ and year $\gamma_{t}$. 

\begin{figure}[h!]
\includegraphics[width = \textwidth]{LP}
\caption{Local Projection Estimates of the Relationship Between TFP Growth and Human Capital}
\label{LP}
{\scriptsize Notes: Data are from the EU KLEMS. The figure presents estimates of the incidence of TFP growth on labor composition or growth rate attributed to labor composition change. Coefficients are for responses of the outcome variables at $t = \{1,2,3,4,5,6\}$ to observed TFP shocks in $t = 0$. TFP shocks are rescaled to have a unit standard deviation. Bands are 95\% and 70\% CI.}
\end{figure}

The estimation result is plotted in Figure \ref{LP}. Coefficients plotted in the figure are the response of outcomes to the TFP shock in $t = 0$. The trend is consistent with the result above. The TFP shock does not affect the labor composition immediately but has a long-run effect on human capital growth. The effect of TFP on human capital gradually increases over time and peaks five years after the technological change. 

A similar pattern can be found using occupational data. When an occupation is more exposed to automation, we should observe a higher growth in skill level. To test the correlation between automation and skill level growth, I use O*NET data and estimate the equation:
\begin{align}
y_{ijt} &= \beta_0 + \beta_1' T_t +\beta_2' T_t \times AOE_{i}+\alpha_0 B_t+\alpha_j' T_t \times D_j +\gamma_j D_j+ \epsilon_{ijt},
\end{align}
where $y_{ijt}$ is the level of skill $j$ for occupation $i$ at time $t$. $T_t$ is a vector of time dummy variables created for each release date $t$. $AOE_{i}$ is the automation occupational exposure for occupation $i$, ranging between 0\% and 100\%. $T_t \times AOE_{i}$ is the interaction between time dummy variables and automation. $B_t$ is a dummy variable controlling for the update in June 2010, $B_t = 1$ if the data is released after June 2010. $D_j$ is a dummy variable created for each skill $j$. The automation effect is estimated while controlling the fixed effect $D_j$ and the common time trend $T_t \times D_j$ for each skill $j$. 

\begin{figure}[h!]
\includegraphics[width = \textwidth]{LV}
\caption{Responses of Skill Levels to Automation}
\label{LV}
{\scriptsize Notes: Data are from the O*NET. The figure presents estimates of the effect of automation exposure on skill level. Coefficients are for skill level responses to automation exposure at each release date ($\beta_2'$). The automation effect is estimated while controlling the fixed effect and the common time trend for each skill. Bands are 95\% CI.}
\end{figure}
$\beta_2'$ is interpreted as the effect of automation on skill level at each release date. The estimation result for $\beta_2'$ is reported in Figure \ref{LV}. The effect of automation on skill level is increasing overtime for all skill groups except systems skills. This result is consistent with the estimation at industry level, workers respond to automation by increasing their skill levels. 

\subsection{Heterogeneous Responses}
The key empirical implication of my model is that skilled and unskilled workers respond differently to automation. First, automation benefits skilled workers more than unskilled workers. Second, skilled workers have a better ability to learn. When an occupation is exposed to automation, skilled workers increase their skill level more than unskilled workers. Using O*NET data, I estimate the equation: 
\begin{align}
\begin{split}
y_{ijt} &= \beta_0 + \beta_1' T_t +\beta_2' T_t \times AOE_{i}+\beta_3' T_t  \times EDUC_{it} +\beta_4' T_t \times AOE_{i} \times EDUC_{it}\\
         &+\alpha_0 B_t+\alpha_j' T_t \times D_j +\gamma_j D_j +\epsilon_{ijt},
\end{split}
\end{align}
where $y_{ijt}$ is the level of skill $j$ for occupation $i$ at time $t$. $T_t$ is a vector of time dummy variables created for each release date $t$. $AOE_{i}$ is the automation occupational exposure for occupation $i$, ranging between 0\% and 100\%. $EDUC_{i}$ is the ratio of skilled workers (college graduates or higher) at occupation $i$, ranging between 0\% and 100\%. $T_t \times AOE_{i}$, $T_t \times EDUC_{i}$ and $T_t \times AOE_{i}\times EDUC_{i}$ are the interaction between time dummy variables and automation or education. I use  $T_t\times EDUC_{it}$ to control for the relocation effect. When unskilled workers get displaced due to automation, the average education level increases and positively affects the average occupational skill level. $B_t$ is a dummy variable controlling for the update in June 2010, $B_t = 1$ if the data is released after June 2010. $D_j$ is a dummy variable created for each skill $j$. The automation effect is estimated while controlling the fixed effect $D_j$ and the common time trend $T_t \times D_j$ for each skill $j$. 

\begin{figure}[h!]
\includegraphics[width = \textwidth]{LV_group1} \\

\includegraphics[width = \textwidth]{LV_group3}
\caption{Heterogeneous Responses of Skill Levels to Automation}
\label{LV_group}
{\scriptsize Notes: Data are from the O*NET. The figure presents estimates of the effect of automation exposure on skill level. Coefficients in the upper panel are for responses of the skill levels to automation exposure at each release date ($\beta_2'$). Coefficients in the lower panel are for the difference between skilled and unskilled workers in skill level responses to automation exposure at each release date ($\beta_4'$).The automation effect is estimated while controlling the relocation effect, and the fixed effect and common time trends for each skill. Bands are 95\% CI.}
\end{figure}

$\beta_2'$ is interpreted as the effect of automation on skill level at each release date. The key coefficient of the regression is $\beta_4'$, which captures the different human capital responses of skilled and unskilled workers facing automation. The estimation result for $\beta_2'$ (upper panel) and $\beta_4'$ (lower panel) are reported in Figure \ref{LV_group}. Contrary to the estimation result in the last subsection, the effect of automation on skill level decreases over time after controlling for the education level of each occupation. Consistent with the predictions of my model, when the workers are more exposed to automation, they have less incentive to invest in their human capital due to the displacement effect. However, as documented in the precious subsection, the overall skill levels increase more as the occupation is more exposed to automation. The increase is contributed mainly by worker relocation and human capital investment of skilled workers. Automation displaces unskilled workers, increases the share of skilled workers at this occupation. Skilled workers acquire new skills adapting to automation and benefit more from technological change. 

Estimation result of $\beta_4'$ shows that skilled and unskilled workers respond differently to exposure to automation. Skilled workers increase their human capital investment more than unskilled workers, especially social and systems skills. The difference between skilled and unskilled workers is increasing overtime, the human capital gap enlarges when workers are exposed to automation. 

The result is consistent with the empirical result of Acemoglu et al. (2020)\nocite{Acemogluetal2020}. Robots adopters experience a decline in the share of production workers but increase the overall employment. They hire more skilled workers and equip them with new skills to take advantage of the technology.  The difference between skilled and unskilled workers could be underestimated using the equation above because of the relocation. If the worker exposed to automation has low skill levels, he is at higher risk of being displaced, and his human capital accumulation slows down more due to joblessness. 

A similar pattern can be found using the direct measure of human capital investment. I use ATES data and calculate the total training hours for each individual spent on ESL, ABE and work experience programs. I only keep employed individuals, so their exposure to automation can be measured based on their occupation. I estimate the equation: 
\begin{align}
y = \beta_0 + \beta_1 AOE + \beta_2' EDUC +\beta_3' AOE \times EDUC + \alpha_0' T  +\alpha_1' \Gamma + \epsilon,
\end{align}
where $y$ represents the total training hours for each individual. $AOE$ is the automation occupational exposure for the worker's occupation. $EDUC$ is a dummy vector to control for the worker's highest degree/credential obtained (Less than high school diploma/High school diploma or equivalent/Associate's degree/Bachelor's degree or higher). $T$ is a vector of year dummy variables created for each survey year. $\Gamma$ is a vector of control variables, including age, sex and ethnicity. 

\begin{table}[h!]
\begin{center}
\scriptsize
\begin{tabular}{lcc} \hline \hline
 & (1) & (2) \\
VARIABLES & \multicolumn{2}{c}{Training hours} \\ \hline
 &  &  \\
High school $\times$ AOE & -4.544*** & -7.138*** \\
 & (0.0672) & (0.0671) \\
Associate $\times$ AOE  & 7.670*** & 4.465*** \\
 & (0.0863) & (0.0862) \\
Bachelor or higher $\times$ AOE & 5.722*** & 3.765*** \\
 & (0.0794) & (0.0793) \\
AOE & -2.314*** & -2.857*** \\
 & (0.0604) & (0.0606) \\
High school  & 2.959*** & 2.438*** \\
 & (0.0325) & (0.0324) \\
Associate & 4.718*** & 4.356*** \\
 & (0.0386) & (0.0386) \\
Bachelor or higher & 10.36*** & 10.16*** \\
 & (0.0342) & (0.0342) \\
Constant & 18.28*** & 36.38*** \\
 & (0.0299) & (0.0344) \\
 &  &  \\
Observations & 495,266,531 & 495,266,531 \\
 R-squared & 0.018 & 0.022 \\ \hline
  Time dummies & Yes & Yes \\
 Control variable & No & Yes \\  \hline
\end{tabular}
\end{center}
\caption{Effect of Automation on Training}
\label{estimation5}
{\scriptsize Notes: Data are from the ATES (Adult Training and Education Survey). The table presents estimates of the incidence of automation and education on training hours (weighted least squares). $AOE$ is the automation occupational exposure for the worker's occupation. $EDUC$ is the highest degree/credential obtained (Less than high school diploma/High school diploma or equivalent/Associate's degree/Bachelor's degree or higher). In the second column, I control for sex, age and ethnicity. Robust standard errors are in parenthesis. ***, **, * denotes statistical significance at 1, 5 and 10 percent levels.}
\end{table}

The key coefficient of the regression is $\beta_3$, which captures the different training investment of skilled and unskilled workers in response to automation. The estimation result is reported in Table \ref{estimation5}. The result is consistent with the occupational skill level change above. Workers working in occupations more exposed to automation invest less in training, and skilled workers invest more in training than unskilled workers. With a 1 percent increase in automation level, workers with bachelor's degree spend 0.1 hour more than workers with only high school degree. I also find similar pattern using NLSY97 (Appendix E).

The empirical findings are consistent with the theoretical implications of my model. Automation decreases human capital investment through the displacement effect, but increases human capital investment through the productivity effect. The responses of skilled and unskilled to automation are significantly different. Automation increases the overall skill level and human capital investment, and the increase is attributed mainly to skilled workers. 

\section{Conclusion}
This paper discusses the adjustments of human capital to a reduction in automation costs. When a technological revolution lowers the automation cost, the automation level increases immediately, which displaces workers and decreases the wage growth rate. A higher automation level improves production factor allocation and increases innovation incentives in the R\&D sector. The innovation rate catches up with the automation rate and encourages human capital investment. A higher human capital growth rate increases the effective labor supply, complementing innovation and substituting automation. With endogenous human capital responses, the economy converges to a balanced growth path with a lower automation level and a higher labor share compared to the scenario with fixed human capital. However, skilled workers respond more than unskilled workers to the technological revolution. Skilled workers benefit more from the productivity effect, while unskilled workers are affected more by the displacement effect. Skilled workers also have more ability to adjust their human capital investment. The uneven responses amplify wage inequality.

Indeed, the data supports the implications of this mechanism. Occupations with higher automation exposure experience higher skill growth rates. This higher growth rate is attributed to worker relocation and human capital accumulation of skilled workers. High automation exposure decreases the human capital investment incentive of unskilled workers. Skilled workers adjust more than unskilled workers in the human capital facing automation exposure. 

Theory and data suggest that the new technological changes are skill-biased and complement skilled workers (Violante, 2018\nocite{Violante2008}). In my model, I assume constant elasticity of substitution between production factors, which may underestimate the uneven effect of automation on different skill groups. The allocation friction of unskilled workers could also contribute to inequality after an increase in automation level. Another extension of my model is to consider multi-dimensional skills; as shown in the empirical section, each skill group responds differently to ongoing technological changes. 

\clearpage
\bibliographystyle{plain}
\bibliography{Growth}
\clearpage

\begin{appendices}

\section{Firm Problem}

\subsubsection*{Final Good Producer}
The competitive final good producers solve the following problem:
\begin{align*}
\max \quad & Y-\int_{N-1}^Np(i)y(i)di, \\
&Y = \tilde{A}\Big(\int_{N-1}^{N}y(i)^{\frac{\sigma-1}{\sigma}}di\Big)^{\frac{\sigma}{\sigma-1}}.
\end{align*}

Take first order condition with respect to $y(i)$:
\begin{align*}
\tilde{A}\Big(\int_{N-1}^{N}y(i)^{\frac{\sigma-1}{\sigma}}di\Big)^{\frac{1}{\sigma-1}}y(i)^{-\frac{1}{\sigma}}-p(i) = 0.
\end{align*}
Then the demand function of each task can be written as:
\begin{align*}
y(i) = \tilde{A}^{\sigma-1}Yp(i)^{-\sigma}.
\end{align*}

\subsubsection*{Task Producer}
The task producers who have adopted machine solves the following problem:
\begin{align*}
\max \quad  p(i)&y(i)-Rk(i)-\psi q(i), \\
&y(i) = q(i)^{\eta}k(i)^{1-\eta}.
\end{align*}
Take first order condition with respect to $k(i)$ and $q(i)$:
\begin{align*}
q(i)\psi &= \eta p(i)y(i), \\
k(i)R &= (1-\eta)p(i)y(i).
\end{align*}
The unit cost of production can be written as:
\begin{align*}
y(i) &= (\frac{\eta p(i)y(i)}{\psi})^{\eta}(\frac{(1-\eta)p(i)y(i)}{R})^{1-\eta}, \\
p(i) &= (\frac{\psi}{\eta})^{\eta} (\frac{R}{1-\eta})^{1-\eta} \\
 	  &= \Psi R^{1-\eta}, \quad \Psi =  (\frac{\psi}{\eta})^{\eta} (\frac{1}{1-\eta})^{1-\eta}.
\end{align*}
Plug in the demand function solved from final good producer problem, the output of the task $p(i)y(i)$ can then be solved: 
\begin{align*}
p(i)y(i) = (\frac{\tilde{A}}{\Psi})^{\sigma-1}YR^{(1-\eta)(1-\sigma)} = Y(\frac{R}{A})^{1-\hat{\sigma}},
\end{align*}
where $\hat{\sigma} = 1-(1-\eta)(1-\sigma)$ and $A = \Big(\frac{\tilde{A}}{\Psi}\Big)^{\frac{\sigma-1}{\hat{\sigma}-1}}  = \Big(\frac{\tilde{A}}{\Psi}\Big)^{\frac{1}{1-\eta}}$. 
The demand of capital for task $i$ is: 
\begin{align*}
k(i) = (1-\eta)A^{\hat{\sigma}-1}YR^{-\hat{\sigma}}.
\end{align*}
Similarly, the unit cost of task if the producers use unskilled or skilled workers can be solved as:
\begin{align*}
p(i) =
\begin{dcases}
\Psi(\frac{W_L}{\gamma_L(i,h_L)})^{1-\eta}, \quad \text{unskilled worker}  \\
\Psi(\frac{W_H}{\gamma_H(i,h_H)})^{1-\eta}, \quad \text{skilled worker}.
\end{dcases}
\end{align*}
The output of the tasks $p(i)y(i)$ are:
\begin{align*}
p(i)y(i) =
\begin{dcases}
Y(\frac{W_L}{A\gamma_L(i,h_L)})^{1-\hat{\sigma}}, \quad \text{unskilled worker}  \\
Y(\frac{W_H}{A\gamma_H(i,h_H)})^{1-\hat{\sigma}}, \quad \text{skilled worker}.
\end{dcases}
\end{align*}
And the demand of unskilled and skilled workers can be solved as: 
\begin{align*}
l(i) &= (1-\eta)A^{\hat{\sigma}-1}\frac{Y}{\gamma_L(i,h_L)}(\frac{W_L}{\gamma_L(i,h_L)})^{-\hat{\sigma}}, \\
h(i) &= (1-\eta)A^{\hat{\sigma}-1}\frac{Y}{\gamma_H(i,h_H)}(\frac{W_H}{\gamma_H(i,h_H)})^{-\hat{\sigma}}.
\end{align*}
 
\section{Proposition Proofs}

\subsubsection*{Factor Allocation}
The research sector always automates the least updated tasks since they are the tasks where machine have less disadvantage. The research sector will develop the patent only if the firm can make more profit by adopting machine and is willing to purchase the patent. As a result, for all the task that are automated ($N-1 \le i \le I$), it must be true that:
\begin{align*}
R^{1-\eta} < \min \{(\frac{W_L}{\gamma_L(i,h_L)})^{1-\eta},(\frac{W_H}{\gamma_H(i,h_H)})^{1-\eta}\},
\end{align*}
and all the firms who have access to automation patent choose to produce the task using machine. 

For the tasks that are not automated, skilled workers always have advantage over more updated tasks. $\frac{W_H}{W_L}$ is increasing in $S$, when $S\to I$, $\frac{W_H}{W_L}\to 0$, when $S\to N$, $\frac{W_H}{W_L}\to \infty$. $ \frac{\gamma_H(S,h_L)}{\gamma_L(S,h_L)}$ is increasing in $S$. There exists a unique cutoff point $S$ such that:
\begin{align*}
\frac{W_H}{W_L} = \frac{\gamma_H(S,h_L)}{\gamma_L(S,h_L)},
\end{align*}
Then, for other tasks that are not automated, we get: 
\begin{align*}
\frac{W_H}{\gamma_H(i,h_L)}\frac{\gamma_L(i,h_L)}{W_L} = e^{(B-B_N)(S-i)}\frac{W_H}{\gamma_H(S,h_L)}\frac{\gamma_L(S,h_L)}{W_L} =  e^{(B-B_N)(S-i)}.
\end{align*}

If $i<S$, then $\frac{W_H}{\gamma_H(i,h_L)}\frac{\gamma_L(i,h_L)}{W_L}>1$, the firm will choose to produce the task using unskilled workers. If $i>S$ then $\frac{W_H}{\gamma_H(i,h_L)}\frac{\gamma_L(i,h_L)}{W_L}<1$, the firm will choose to produce the task using skilled workers.

\subsubsection*{Aggregate Production Function}
At equilibrium, the market clearing condition needs to be satisfied:
\begin{align*}
K &= \int_{N-1}^I k(i)di=  \int_{N-1}^I(1-\eta)A^{\hat{\sigma}-1}YR^{-\hat{\sigma}}di, \\
L_L &= \int_{I}^S l_L(i)di=  \int_{N-1}^I (1-\eta)A^{\hat{\sigma}-1}\frac{Y}{\gamma_L(i,h_L)}(\frac{W_L}{\gamma_L(i,h_L)})^{-\hat{\sigma}}di, \\
L_H &=\int_{S}^N l_H(i)di=  \int_{N-1}^I (1-\eta)A^{\hat{\sigma}-1}\frac{Y}{\gamma_H(i,h_H)}(\frac{W_H}{\gamma_H(i,h_H)})^{-\hat{\sigma}} di.
\end{align*}
Then the factor prices could be solved using market clearing condition: 
\begin{align*}
R &=A\Gamma_K \Big(\frac{(1-\eta)Y}{A\Gamma_K K}\Big)^{\frac{1}{\hat{\sigma}}}, \quad \Gamma_K= (I-N+1)^{\frac{1}{\hat{\sigma}-1}},  \\
W_L &= A\Gamma_L\Big(\frac{(1-\eta)Y}{A\Gamma_LL_L}\Big)^{\frac{1}{\hat{\sigma}}}, \quad \Gamma_L=\Big(\frac{\gamma_L(S,h_L)^{\hat{\sigma}-1}-\gamma_L(I,h_L)^{\hat{\sigma}-1}}{B_L(\hat{\sigma}-1)}\Big)^{\frac{1}{\hat{\sigma}-1}}, \\
W_H &=A\Gamma_H\Big(\frac{(1-\eta)Y}{A\Gamma_HL_H}\Big)^{\frac{1}{\hat{\sigma}}}, \quad \Gamma_H = \Big(\frac{\gamma_H(N,h_H)^{\hat{\sigma}-1}-\gamma_H(S,h_H)^{\hat{\sigma}-1}}{B(\hat{\sigma}-1)}\Big)^{\frac{1}{\hat{\sigma}-1}}. 
\end{align*}

Plug in the factor price and output for each task:
\begin{align*}
Y &= \int_{N-1}^N p(i)y(i) di \\
	&= \int_{N-1}^I Y(\frac{R}{A})^{1-\hat{\sigma}} di + \int_I^S Y(\frac{W_L}{A\gamma_L(i,h_L)})^{1-\hat{\sigma}} di  + \int_S^N Y(\frac{W_H}{A\gamma_H(i,h_H)})^{1-\hat{\sigma}} di, \\
\Big(\frac{(1-\eta)Y}{A} \Big)^{\frac{\hat{\sigma}-1}{\hat{\sigma}}} &= \int_{N-1}^I \frac{(\Gamma_K K)^{\frac{\hat{\sigma}-1}{\hat{\sigma}}}}{\Gamma_K^{\hat{\sigma}-1}} di + \int_I^S \frac{(\Gamma_LL_L \gamma_L(i,h_L))^{\frac{\hat{\sigma}-1}{\hat{\sigma}}}}{\Gamma_L^{\hat{\sigma}-1}} di  + \int_S^N \frac{(\Gamma_HL_H \gamma_H(i,h_H))^{\frac{\hat{\sigma}-1}{\hat{\sigma}}}}{\Gamma_H^{\hat{\sigma}-1}} di.
 \end{align*}
The final good could be written as 
\begin{align*}
Y = \frac{A}{1-\eta}\Big((\Gamma_K K)^{\frac{\hat{\sigma}-1}{\hat{\sigma}}}+(\Gamma_LL_L)^{\frac{\hat{\sigma}-1}{\hat{\sigma}}}+(\Gamma_HL_H)^{\frac{\hat{\sigma}-1}{\hat{\sigma}}}\Big)^{\frac{\hat{\sigma}}{\hat{\sigma}-1}}.
 \end{align*}

\subsubsection*{Human Capital Growth and Automation Level}
The long run interest rate needs to satisfy: 
\begin{align*}
r = \rho+\theta g = \rho+\theta(Bg_N+bg_{hH}).
\end{align*}
Higher human capital growth rate increases the long run interest rate, which increases the innovation patent value and decreases the automation patent value. The automation level needs to decrease in order to have: 
\begin{align*}
\frac{p_N}{p_I} = (\frac{\mu_N}{\mu_I})^{\frac{1}{\lambda}}.
\end{align*} 
 
\subsubsection*{Human Capital and Technology Growth Rate}
On the balanced growth path, the human capital investment needs to satisfy the following equation: 
\begin{align*}
(1-\theta)\frac{b}{\mu_hj}(1-l_j)^{\alpha_j}+\alpha\frac{b}{\mu_hj}l_j(1-l_j)^{\alpha_j-1} = \rho+(\theta-1)Bg_N.
\end{align*}
Define $f(l)$ as the left hand side of the equation, I can take derivative of $f(l)$ as:
\begin{align*}
f'(l) = \alpha_j\frac{b}{\mu_hj}(1-l_j)^{\alpha_j-1}\big(\theta+(1-\alpha_j)\frac{l_j}{1-l_j}\big)>0.
\end{align*}
If $\theta-1>0$, the labor supply is increasing in $g_N$; If $\theta-1<0$, the labor supply is decreasing in $g_N$. 

\subsubsection*{Long-run comparative statistics}
On the balanced growth path, the value of patent needs to satisfy: 
\begin{align*}
\frac{p_N}{p_I} = (\frac{\mu_N}{\mu_I})^{\frac{1}{\lambda}}.
\end{align*} 
The left hand side is increasing in the automation level. Higher automation level decrease the demand for labor and decreases the wage level. Lower wage level increases the innovation patent value $p_N$ and decreases the automation patent value $p_I$. The right hand side is decreasing in the cost of automation. When the automation cost decreases, the right hand side decreases, the automation level must increase. 

The number of scientist working on the innovation sector must satisfy: 
\begin{align*}
 \frac{\lambda\epsilon_N(t)^{\lambda-1}}{\mu_N}p_N(t) = \omega_H(t).
\end{align*} 
Higher automation level increases $p_N$ and increases the number of scientists working in the innovation sector. The innovation rate is higher. 

On the balanced growth path, the human capital of skilled and unskilled workers need to grow at the same rate $g_{hH} = g_{hL} = g_h$. Using the policy function for human capital investment, I can get:
\begin{align*}
\alpha_H((\mu_{hH}g_{hH})^{-\frac{1}{\alpha_H}}-1) = \alpha_L((\mu_{hL}g_{hL})^{-\frac{1}{\alpha_L}}-1).
\end{align*}

Then I can derive the ratio of human capital investment cost and human capital gap as:
\begin{align*}
\log(\frac{\mu_{hH}}{\mu_{hL}}) &= (\frac{\alpha_H}{\alpha_L}-1)\log(\mu_{hL}g_h)-\alpha_H\log((\frac{\alpha_H}{\alpha_L}-1)(\mu_{hL}g_h)^{\frac{1}{\alpha_L}}+1)-\log(\frac{\alpha_H}{\alpha_L}), \\
h_{HL}-h_{HL}^*&= -\frac{\alpha_H}{\lambda_h}\log \Big(\frac{\alpha_L(\mu_{hL}g_h)^{-\frac{1}{\alpha_L}}+\alpha_H-\alpha_L}{\alpha_H(\mu_{hL}g_h)^{-\frac{1}{\alpha_H}}}\Big).
\end{align*}
Take derivatives on $g_h$, I can get:
\begin{align*}
dh_{HL} = \frac{1}{\lambda_h}\frac{(\alpha_H-\alpha_L)l_L}{(1-l_L)\alpha_H+l_L\alpha_L}d\log g_h.
\end{align*}

\section{Calibration}
To calibrate the parameters internally, I need to find the relevant moments. 
Discount rate $\rho$ can calibrated using the long-run interest rate and growth rate:
\begin{align*}
r = \rho+\theta g.
\end{align*}
Patent share $\eta$ is calibrated using the ratio of R\&D investment and GDP:
\begin{align*}
\frac{W_H(\epsilon_N+\epsilon_I)}{Y} = (\frac{W_H}{Y})^{\frac{\lambda}{\lambda-1}}\Big((\frac{1}{\lambda}\frac{\mu_N}{p_N})^{\frac{1}{\lambda-1}}+(\frac{1}{\lambda}\frac{\mu_I}{p_I})^{\frac{1}{\lambda-1}}\Big) \propto \eta^{\frac{1}{1-\lambda}}.
\end{align*}
Cost of innovation needs to set to match the innovation growth rate: 
\begin{align*}
g_N = (\frac{1}{\mu_N})^{\frac{1}{1-\lambda}}(\frac{\lambda P_N}{W_H})^{\frac{\lambda}{1-\lambda}} \propto (\frac{1}{\mu_N})^{\frac{1}{1-\lambda}}.
\end{align*}
Capital productivity $A$ and cost of automation are calibrated using short run interest rate and capital share:
\begin{align*}
R &= AH\Big(\frac{s_K}{1-\eta}\Big)^{\frac{1}{1-\hat{\sigma}}} \propto A, \\
\frac{P_I}{P_N} &=(\frac{\mu_I}{\mu_N})^{\frac{1}{\lambda}} \propto \mu_I^{\frac{1}{\lambda}}.
\end{align*}

Skill comparative advantage $B_H$ and training cost elasticity $\lambda_h$ is calibrated using wage premium in 1980 and 2005: 
\begin{align*}
\omega &= (\frac{\Gamma_H}{\Gamma_L})^{\frac{{\hat{\sigma}}-1}{\hat{\sigma}}}, \\
d \ln \omega &=(B-B_L)d\tilde{S}+bdh_{HL},  \\
dh_{HL} &= \frac{1}{\lambda_h}\frac{(1-\frac{\alpha_L}{\alpha_H})l_L}{1-(1-\frac{\alpha_L}{\alpha_H})l_L}d\log(g_h) \propto \frac{1}{\lambda_h}.
\end{align*}

Training law of motion $\alpha_H$ and $\alpha_L$ are calibrated using the change of training time:
\begin{align*}
d\log(1-l_H) &= \frac{1}{(1-l_L)\alpha_H+l_L\alpha_L}d\log(g_h), \\
d\log(1-l_L) &= \frac{1}{\alpha_L}d\log(g_h).
\end{align*}

Size of technology wave is calibrated to match the change of labor share:
\begin{align*}
\frac{P_I}{P_N} =(\frac{\mu_I}{\mu_N})^{\frac{1}{\lambda}} \propto \mu_I^{\frac{1}{\lambda}}.
\end{align*}

\section{Transition Path Algorithm}

To solve the transition path numerically between two balanced growth path. I firstly solve the BGP before and after political or technological shock, and use them as the starting and end point. 

\begin{itemize}
\item[(1)] Start with the initial guess of capital stock $\{k_0(t)\}$, labor supply $\{L_{H0}(t), L_{L0}(t)\}$, automation level $\{\tilde{I}_0(t)\}$, human capital gap $\{h_{HL0}(t)\}$ and growth rates $\{g_{N0}(t),g_{hH0}(t),g_{hL0}(t)\}$. 


\item[(2)] By solving firm's problem, I can get prices $\{r(t),\omega_H(t),\omega_L(t)\}$, capital and labor share $\{s_K(t),s_H(t),s_L(t)\}$:
\begin{align*}
r &= A \Gamma_K(\frac{(1-\eta)y}{A\Gamma_Kk})^{\frac{1}{\hat{\sigma}}}-\delta \\
\omega_H &= A \tilde{\Gamma}_H(\frac{(1-\eta)y}{A\tilde{\Gamma}_HL_H})^{\frac{1}{\hat{\sigma}}} \\
\omega_L &= A \frac{\tilde{\Gamma}_L}{\gamma_{HL}}(\frac{(1-\eta)y}{A\frac{\tilde{\Gamma}_L}{\gamma_{HL}}L_L})^{\frac{1}{\hat{\sigma}}} \\
s_K &= (1-\eta)(\frac{R}{A\Gamma_K})^{1-\hat{\sigma}} \\
s_H &= (1-\eta)(\frac{\omega_H}{A \tilde{\Gamma}_H})^{1-\hat{\sigma}} \\
s_L &= (1-\eta)(\frac{\omega_L}{A \frac{\tilde{\Gamma}_L}{\gamma_{HL}}})^{1-\hat{\sigma}} 
\end{align*}
and task allocation $\{\tilde{S}(t)\}$: 
\begin{align*}
\frac{\omega_H}{\gamma_H(h_L,S)} &= \frac{\omega_L}{\gamma_L(h_L,S)}
\end{align*}

\item[(3)] Given factor prices $\{r(t),\omega_H(t),\omega_L(t)\}$ and growth rates $\{g_{N0}(t),g_{hH0}(t),g_{hL0}(t)\}$, I can solve consumption $\{c_H(t),c_L(t)\}$ and capital $\{k_H(t),k_L(t)\}$ by solving household's problem:
\begin{align*}
g &= B_Hg_{N0}+bg_{hH0} \\
\frac{\dot{c}_H}{c_H} &= \frac{\dot{c}_L}{c_L} = \frac{r-\rho}{\theta}-g \\
\dot{k}_H &= (r-g)k_H+\omega_HL_H+\pi-c_H \\
\dot{k}_L &=  (r-g)k_L+\omega_L L_L-c_L
\end{align*}
then solve the new capital stock $\{k_1(t)\}$:
\begin{align*}
\dot{k} &= \dot{k}_H+ \dot{k}_L
\end{align*}

\item[(4)] Given factor prices $\{r(t),\omega_H(t),\omega_L(t)\}$ and growth rates $\{g_{N0}(t),g_{hH0}(t),g_{hL0}(t)\}$, I can solve the patent value of R\&D sector:
\begin{align*}
p_N &= (1-\tau_N)v_N(N)-(1-\tau_I)v_I\\
p_I &= (1-\tau_I)v_I-(1-\tau_N)v_N(I) 
\end{align*}
Where $V_{NH}$, $V_{NL}$ and $V_I$ can be solved recursively given factor prices and growth rates. 


\item[(5)] Given factor prices $\{r(t),\omega_H(t),\omega_L(t)\}$, patent value $\{p_N(t),p_I(t)\}$ I can solve human capital investment $\{g_{hH}(t), g_{hL}(t)\}$ using HH's problem:
\begin{align*}
\frac{b}{\mu_{hH}}\alpha_Hl_H(1-l_H)^{\alpha_H-1}+bg_{hH}+g_{WH} &= (1+\tau_{hH})r\\
\frac{b}{\mu_{hL}}\alpha_Ll_L(1-l_L)^{\alpha_L-1}+bg_{hL}+g_{WL} &= (1+\tau_{hL})r
\end{align*}

and growth rates $\{g_{N1}(t), g_{I1}(t)\}$ and labor supply $\{L_{H1}(t), L_{L1}(t)\}$ using research sector's problem: 
\begin{align*}
\frac{p_N}{\mu_N} &= \frac{\omega_H}{y}  \\
\frac{p_I}{\mu_I} &= \frac{\omega_H}{y}  \\
L_H &=\epsilon_H l_H-\mu_Ng_N-\mu_Ig_I \\
L_L &=\epsilon_L l_L
\end{align*}

\item[(6)] Given growth rate $\{g_{N1}(t),g_I(t),g_{hH1}(t),g_{hL1}(t)\}$, I can solve the new automation level $\{\tilde{I}_1(t)\}$ and human capital gap $\{h_{HL1}(t)\}$ using the law of motion: 
\begin{align*}
\dot{\tilde{I}} &= g_I- g_N \\
\dot{h}_{HL} &= b (g_{hH}-g_{hL})
\end{align*}

\item[(7)] Update capital stock $\{k_0(t)\}$, labor supply $\{L_{H0}(t), L_{L0}(t)\}$, automation level $\{\tilde{I}_0(t)\}$, human capital gap $\{h_{HL0}(t)\}$ and growth rates $\{g_{N0}(t),g_{hH0}(t),g_{hL0}(t)\}$, and go back to 2 until converges. 

\end{itemize}

\section{Empirical Appendix}
\subsection{Changes in Skill Level and Human Capital Investment}
\begin{table}[h!]
\begin{center}
\scriptsize
\begin{tabular}{l|cccc} 
\hline \hline
& \multicolumn{4}{c}{$100 \times$Annualized log change} \\
Industry & 1995 & 2000 & 2005 & 2010  \\ \hline
Accommodation and food service activities & -4.621 & 5.880 & 3.514 & 6.990 \\
Agriculture, forestry and fishing & 7.564 & 8.061 & 5.722 & 2.950 \\ 
Arts, entertainment and recreation & 5.433 & 4.225 & 3.221 & 2.307 \\
Construction & 2.375 & 2.481 & 2.821 & 3.721 \\
Education & 0.836 & 0.263 & 1.071 & 1.127 \\
Electricity, gas, steam and air conditioning supply & -0.389 & 1.290 & 3.568 & 3.925 \\
Financial and insurance activities & 2.248 & 1.412 & 1.669 & 3.753 \\
Health and social work & 2.564 & 0.822 & 1.937 & 2.432 \\
Information and communication & 3.194 & 1.934 & 3.660 & 2.844 \\
Mining and quarrying & 6.689 & 3.550 & 4.495 & 6.311 \\
Other service activities & 3.793 & 4.261 & 3.511 & 3.300 \\
Professional, scientific, technical, administrative & \multirow{2}{*}{1.491} & \multirow{2}{*}{1.358} & \multirow{2}{*}{1.335} & \multirow{2}{*}{2.495} \\
and support service activities &&&&\\
Public administration and defense,  & \multirow{2}{*}{2.186} & \multirow{2}{*}{2.553} & \multirow{2}{*}{2.808} & \multirow{2}{*}{2.806}\\
compulsory social security &&&&\\
Real estate activities & 3.613 & 2.000 & 3.091 & 2.656 \\
Total manufacturing & 3.884 & 2.619 & 4.248 & 3.707 \\
Transportation and storage & 2.991 & 2.305 & 6.175 & 4.027 \\
Water supply, sewerage,  & \multirow{2}{*}{3.532} & \multirow{2}{*}{2.446} & \multirow{2}{*}{1.437} & \multirow{2}{*}{4.999}  \\
waste management and remediation activities &&&&\\
Wholesale and retail trade & \multirow{2}{*}{3.984} & \multirow{2}{*}{2.603} & \multirow{2}{*}{2.770} & \multirow{2}{*}{4.607} \\
repair of motor vehicles and motorcycles &&&&\\
Total & 2.854  & 2.781 & 3.172  & 3.609 \\\hline
\end{tabular}
\end{center}
\caption{Trends in Skilled Worker Share by Industry}
\label{Industry_trend}
{\scriptsize Notes: Data are from the EU KLEMS (EU level analysis of capital, labor, energy, materials and service inputs). I calculate the average change of skilled worker shares for the following periods: 1995-1999, 2000-2004, 2005-2009, and 2010-2014.}
\end{table}

Table \ref{Industry_trend} summarizes the change in the share of skilled workers (college graduates or higher) by industry. Between 1995 and 2015, the aggregate share of skilled workers increased. There are significant variations between industries; industries using more unskilled workers like agriculture and manufacturing experienced more labor composition change, while industries using more skilled workers like education and health experienced less.

\begin{figure}[h!]
\includegraphics[width = \textwidth]{trend}
\caption{Trends in Skill Level and Importance}
\label{trend}
{\scriptsize Notes: Data are from the O*NET (Occupational Information Network). I calculate the average skill level or importance for the seven skill groups. The x-axis is the release date of the data; the y-axis is the change in skill level or importance compared with the first release. }
\end{figure}

Figure \ref{trend} summarizes the changes in importance and level for each skill group. Between 2000 and 2020, the trends in level and importance are very similar. The main skill growth happened between 2002 and 2010, and only minor changes were made after the major update in 2010. The level and importance of social skills have grown significantly faster than other skills, while only technical skills are lower and less important than before. 

\begin{figure}[h!]
\includegraphics[width = \textwidth]{LV_trend1}
\caption{Trends in Skill Level Facing Different Automation Exposure}
\label{LV_trend1}
{\scriptsize Notes: Data are from the O*NET (Occupational Information Network). I partition the occupations into groups with low (under the 33rd percentiles), medium (between the 33rd and 66th percentiles) and high (above the 66th percentiles) automation exposures. Then I calculate the average skill level or importance for the seven skill groups. The x-axis is the release date of the data; the y-axis is the change in skill level compared with the first release. }
\end{figure}

To see if automation exposure affects the human capital accumulation as predicted in the model, I partition the occupations into groups with low (under the 33rd percentiles), medium (between the 33rd and 66th percentiles) and high (above 66th percentiles) automation exposures. To do so, I use the automation occupational exposure measure constructed by Webb (2019)\nocite{Webb2019}. He uses the overlap between the text of job task descriptions (O*NET) and the text of patents (Google Patents Public Data) to measure the exposure of tasks to automation. The measure ranges from 0\% to 100\%, indicating how likely the labor is to be displaced in each occupation. Figure \ref{LV_trend1} summarizes the change in level for each skill group by automation exposure. The skill levels of all the skill groups increase significantly more when the occupation is more exposed to automation, especially process, social and complex problem solving skills. 

\begin{figure}[h!]
\includegraphics[width = \textwidth]{LV_trend2}
\caption{Trends in Skill Level with Different Education level}
\label{LV_trend2}
{\scriptsize Notes: Data are from the O*NET (Occupational Information Network). I partition the occupations into groups with low (under the 33rd percentiles), medium (between the 33rd and 66 percentiles) and high (above the 66th percentiles) education. Then I calculate the average skill level or importance for the seven skill groups. The x-axis is the release date of the data; the y-axis is the change in skill level compared with the first release. }
\end{figure}

Another important empirical implication of my model is the heterogeneous responses of different skill groups. The education level is measured by the share of skilled workers working in the occupation. O*NET provides detailed information on workers' education levels for each occupation. I construct the ratio of workers with a college degree or higher as the empirical analog of the share of skilled workers. I partition the occupations into groups with low (under the 33rd percentiles), medium (between the 33rd and 66th percentiles) and high (above 66th percentiles) education levels. Figure \ref{LV_trend2} summarizes the change in level for each skill group with different education levels. The skill levels of the occupation grow significantly faster as more skilled workers work in this occupation, except for technical skills. 

\subsection{Effects of Automation on Human Capital Investment}
Based on the model, after a reduction of automation cost, both automation and innovation rates increase. In the long run, workers respond to the change in technology by investing more in their human capital. As a result, we should observe an overall increase in human capital or an improvement in labor composition (higher ratio of skilled workers). To test the correlation between automation and human capital growth, I use EU KLEMS data and estimate the equation:
\begin{align}
\begin{split}
 \Delta \ln Y_{i c, t-1} &=\beta_{0}+\beta_{1} \Delta \ln TFP_{i, c \neq c(i), t-k} +\beta_{2} \Delta \ln TFP_{i, c \neq c(i), t-k-1}+\beta_{3} \Delta \ln Y_{i c, t-2}\\
 &+\Delta L_{i c, t-1}+\alpha_{c}+\gamma_{t}+\epsilon_{ict}.
 \end{split}
\end{align}
I regress the outcome variable $\Delta \ln Y_{i c, t-1}$ on the TFP growth rate $\Delta \ln TFP_{i, c \neq c(i), t-k}$ with time leg $k \in \{1,2,3,4,5\}$ while controlling the lagged values of both TFP growth $\Delta \ln TFP_{i, c \neq c(i), t-k-1}$ and of outcome variable growth $\Delta \ln Y_{ic, t-2}$. To interpret $\beta_{1}$ as the effect of automation on human capital, not a result of relocation, I control for the change of employment $\Delta L_{i c, t-1}$. I also control for the fixed effect of country $\alpha_{c}$ and year $\gamma_{t}$. To overcome the endogeneity between TFP and labor composition, I follow Autor and Salomons (2018)\nocite{AutorSalomons2018}. I construct industry-level TFP growth for each industry-country pair as the leave-out mean of industry-level TFP growth in all other countries, then rescaled it to have a unit standard deviation. The coefficient $\beta_{1}$ measures the impact of TFP growth on outcome variables with different lags. 

\begin{table}[h!]
\begin{center}
\scriptsize
\begin{tabular}{lcccccc} \hline \hline
 & (1) & (2) & (3) & (4) & (5) & (6)\\
VARIABLES & \multicolumn{6}{c}{$100 \times$Annualized log change in share of skilled workers} \\ \hline
 &  &  &  &  &  &  \\
TFP Growth & -0.219 &  &  &  &  &  \\
 & (0.311) &  &  &  &  &  \\
TFP Growth (lag 1) &  & 0.339 &  &  &  &  \\
 &  & (0.252) &  &  &  &  \\
TFP Growth (lag 2) &  &  & 0.310 &  &  &  \\
 &  &  & (0.251) &  &  &  \\
TFP Growth (lag 3) &  &  &  & -0.253 &  &  \\
 &  &  &  & (0.257) &  &  \\
TFP Growth (lag 4) &  &  &  &  & 0.550** &  \\
 &  &  &  &  & (0.271) &  \\
TFP Growth (lag 5) &  &  &  &  &  & -0.412 \\
 &  &  &  &  &  & (0.282) \\
Constant & 4.305 & 4.655** & 5.924*** & 7.012*** & 7.012*** & 9.188*** \\
 & (2.770) & (1.971) & (1.963) & (1.987) & (2.033) & (2.090) \\
 &  &  &  &  &  &  \\
Observations & 2,301 & 2,657 & 2,325 & 1,993 & 1,661 & 1,330 \\
 R-squared & 0.120 & 0.123 & 0.135 & 0.151 & 0.178 & 0.223 \\ \hline
\end{tabular}
\end{center}
\caption{TFP Effect on High Educated Labor Share}
\label{estimation1}
{\scriptsize Notes: Data are from EU KLEMS. The table presents estimates of the incidence of TFP growth on labor composition change. Coefficients are for observed TFP shocks in $t = \{0,-1,-2,-3,-4,-5\}$, rescaled to have a unit standard deviation. Robust standard errors are in parenthesis. ***, **, * denotes statistical significance at 1, 5 and 10 percent levels.}
\end{table}

\begin{table}[h!]
\begin{center}
\scriptsize
\begin{tabular}{lcccccc} \hline \hline
 & (1) & (2) & (3) & (4) & (5) & (6)\\
VARIABLES & \multicolumn{6}{c}{Labor composition growth} \\ \hline
 &  &  &  &  &  &  \\
TFP Growth & -0.0787 &  &  &  &  &  \\
 & (0.0739) &  &  &  &  &  \\
TFP Growth (lag 1) &  & 0.0997 &  &  &  &  \\
 &  & (0.0806) &  &  &  &  \\
TFP Growth (lag 2) &  &  & 0.171** &  &  &  \\
 &  &  & (0.0864) &  &  &  \\
TFP Growth (lag 3) &  &  &  & 0.107 &  &  \\
 &  &  &  & (0.0941) &  &  \\
TFP Growth (lag 4) &  &  &  &  & 0.281** &  \\
 &  &  &  &  & (0.109) &  \\
TFP Growth (lag 5) &  &  &  &  &  & 0.0879 \\
 &  &  &  &  &  & (0.121) \\
Constant & -0.817 & -0.924 & -1.093 & -1.301 & -1.389 & -1.610 \\
 & (0.726) & (0.752) & (0.800) & (0.850) & (0.933) & (0.997) \\
 &  &  &  &  &  &  \\
Observations & 2,597 & 2,300 & 2,003 & 1,706 & 1,409 & 1,113 \\
 R-squared & 0.193 & 0.189 & 0.195 & 0.218 & 0.221 & 0.280 \\ \hline
\end{tabular}
\end{center}
\caption{TFP Effect on Labor Composition Growth}
\label{estimation2}
{\scriptsize Notes: Data are from the EU KLEMS. The table presents estimates of the incidence of TFP growth on growth rate attributed to labor composition change. Coefficients are for observed TFP shocks in $t = \{0,-1,-2,-3,-4,-5\}$, rescaled to have a unit standard deviation. Robust standard errors are in parenthesis. ***, **, * denotes statistical significance at 1, 5 and 10 percent levels.}
\end{table}

To measure human capital growth, I use two related variables: the change in the share of skilled workers (college graduates) and the growth rate attributed to labor composition change. The estimation result is reported in Table \ref{estimation1} and \ref{estimation2}. The coefficient $\beta_{1}$ is not significant except when $k=4$. The TFP growth does not immediately impact the labor composition but has a positive effect with a four-year lag. A one standard deviation positive TFP shock increases the share of skilled workers by 0.55\% and increases the growth rate attributed to labor composition change by 0.28\% after four years. 

\subsection{Heterogeneous Responses}
Using NLSY97 data, I construct the cumulative training each worker has spent after entering the labor force.  I estimate the equation: 
\begin{align}
y = \beta_0 + \beta_1 AOE + \beta_2' SKILL +\beta_3' AOE \times SKILL  +\alpha' \Gamma + \epsilon,
\end{align}
where $AOE$ is worker's lifetime automation exposure, $SKILL$ is worker's skill group (1 if the worker has a bachelor's degree, and 0 otherwise). I also control for the worker's characteristics $\Gamma$. 
\begin{table}[h!]
\begin{center}
\scriptsize
\begin{tabular}{lcc} \hline \hline
 & (1) & (2) \\
VARIABLES & \multicolumn{2}{c}{Cumulative training time} \\ \hline
 &  &  \\
SKILL $\times$ AOE & 0.0156* & 0.0184** \\
 & (0.00898) & (0.00898) \\
AOE & -0.0725*** & -0.0814*** \\
 & (0.00333) & (0.00345) \\
SKILL & 0.426 & 0.201 \\
 & (0.340) & (0.341) \\
Constant & 12.32*** & 14.51*** \\
 & (0.176) & (0.308) \\
 &  &  \\
Observations & 93,142 & 93,142 \\
 R-squared & 0.008 & 0.009 \\ \hline
  Control variable & No & Yes \\  \hline
 \end{tabular}
\end{center}
\caption{Effect of Automation on Training}
\label{estimation6}
{\scriptsize Notes: Data are from the Data are from the NLSY97 (National Longitudinal Survey of Youth 1997 Waves). The table presents estimates of the incidence of automation and education on cumulative training length. $AOE$ is the lifetime automation exposure of the worker. $SKILL$ is the worker's skill group (1 if the worker has a bachelor's degree, and 0 otherwise). In the second column, I control for sex, age and ethnicity. Robust standard errors are in parenthesis. ***, **, * denotes statistical significance at 1, 5 and 10 percent levels.}
\end{table}
\end{appendices}

The estimation result is reported in Table \ref{estimation6}, it is consistent with the result using ATES. 

\end{document}
